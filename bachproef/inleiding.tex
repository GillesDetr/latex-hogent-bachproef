%%=============================================================================
%% Inleiding
%%=============================================================================

\chapter{\IfLanguageName{dutch}{Inleiding}{Introduction}}%
\label{ch:inleiding}

%% De inleiding moet de lezer net genoeg informatie verschaffen om het onderwerp te begrijpen en in te zien waarom de onderzoeksvraag de moeite waard is om te onderzoeken. In de inleiding ga je literatuurverwijzingen beperken, zodat de tekst vlot leesbaar blijft. Je kan de inleiding verder onderverdelen in secties als dit de tekst verduidelijkt. Zaken die aan bod kunnen komen in de inleiding~\autocite{Pollefliet2011}:

%%\begin{itemize}
%%  \item context, achtergrond
 %% \item afbakenen van het onderwerp
  %%\item verantwoording van het onderwerp, methodologie
  %%\item probleemstelling
  %%\item onderzoeksdoelstelling
  %%\item onderzoeksvraag
  %%\item \ldots
%%\end{itemize}


\section{\IfLanguageName{dutch}{Probleemstelling}{Problem Statement}}%
\label{sec:probleemstelling}

Het patchmanagement in ERP-systemen brengt unieke uitdagingen met zich mee, met name op het gebied van communicatie en implementatie. Deze problematiek richt zich specifiek op ERP-Beheerders en alle stakeholders betrokken bij de patchmanagement.

\section{\IfLanguageName{dutch}{Onderzoeksvraag}{Research question}}%
\label{sec:onderzoeksvraag}

Hoe kunnen we de patch implementatie in ERP systemen optimaliseren en de communicatie met stakeholders verbeteren, binnen het patchmanagement proces?

\section{\IfLanguageName{dutch}{Onderzoeksdoelstelling}{Research objective}}%
\label{sec:onderzoeksdoelstelling}

Het doel van deze bachelorproef is om praktische inzichten en aanbevelingen te bieden voor het implementeren van patchmanagement in ERP-systemen. Het succes wordt beoordeeld aan de hand van de bruikbaarheid en haalbaarheid van de voorgestelde oplossingen voor de doelgroep van ERP beheerders en stakeholders.

\section{\IfLanguageName{dutch}{Opzet van deze bachelorproef}{Structure of this bachelor thesis}}%
\label{sec:opzet-bachelorproef}

De rest van deze bachelorproef is als volgt opgebouwd:

% Het is gebruikelijk aan het einde van de inleiding een overzicht te
% geven van de opbouw van de rest van de tekst. Deze sectie bevat al een aanzet
% die je kan aanvullen/aanpassen in functie van je eigen tekst.

De rest van deze bachelorproef is als volgt opgebouwd:

In Hoofdstuk~\ref{ch:stand-van-zaken} wordt een overzicht gegeven van de stand van zaken binnen het onderzoeksdomein, op basis van een literatuurstudie.

In Hoofdstuk~\ref{ch:methodologie} wordt de methodologie toegelicht en worden de gebruikte onderzoekstechnieken besproken om een antwoord te kunnen formuleren op de onderzoeksvragen.

In Hoofdstuk~\ref{ch:resultaten} worden de resultaten van het onderzoek gepresenteerd, waarbij praktische inzichten en aanbevelingen worden gegeven voor het implementeren van patchmanagement in cloudgebaseerde ERP-systemen.

In Hoofdstuk~\ref{ch:conclusie} wordt de conclusie gegeven en wordt een antwoord geformuleerd op de onderzoeksvragen. 
% TODO: Vul hier aan voor je eigen hoofstukken, één of twee zinnen per hoofdstuk

In Hoofdstuk~\ref{ch:verdere hoofdstukken} (om niet te vergeten) vul hier verdere hoofdstukken aan.