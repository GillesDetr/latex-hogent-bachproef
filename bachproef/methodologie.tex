%%=============================================================================
%% Methodologie
%%=============================================================================

\chapter{\IfLanguageName{dutch}{Methodologie}{Methodology}}%
\label{ch:methodologie}

%% TODO: In dit hoofstuk geef je een korte toelichting over hoe je te werk bent
%% gegaan. Verdeel je onderzoek in grote fasen, en licht in elke fase toe wat
%% de doelstelling was, welke deliverables daar uit gekomen zijn, en welke
%% onderzoeksmethoden je daarbij toegepast hebt. Verantwoord waarom je
%% op deze manier te werk gegaan bent.
%% 
%% Voorbeelden van zulke fasen zijn: literatuurstudie, opstellen van een
%% requirements-analyse, opstellen long-list (bij vergelijkende studie),
%% selectie van geschikte tools (bij vergelijkende studie, "short-list"),
%% opzetten testopstelling/PoC, uitvoeren testen en verzamelen
%% van resultaten, analyse van resultaten, ...
%%
%% !!!!! LET OP !!!!!
%%
%% Het is uitdrukkelijk NIET de bedoeling dat je het grootste deel van de corpus
%% van je bachelorproef in dit hoofstuk verwerkt! Dit hoofdstuk is eerder een
%% kort overzicht van je plan van aanpak.
%%
%% Maak voor elke fase (behalve het literatuuronderzoek) een NIEUW HOOFDSTUK aan
%% en geef het een gepaste titel.

\subsection{Fase 1: Literatuuronderzoek}
Fase 1 omvat een grondige literatuurverkenning om de huidige staat van ERP-patchmanagement bij managed service provider te doorgronden. In dit kader worden relevante bronnen ontleed, waaronder academische publicaties en vakliteratuur. De kern ligt op het verdiepen van het begrip binnen patchmanagement, het opsporen van bestaande procedures voor patchmanagement en de erkenning van de uitdagingen en succescomponenten. De duur van deze fase wordt geschat op twee weken.
\subsection{Fase 2: Gevalsstudies}
De tweede fase omvat het uitdiepen van specifieke gevallen door middel van een casusonderzoek bij een managed service provider. De analyse zal zich richten op hoe de onderneming het patchmanagementproces beheert. Dit proces zal schematisch in kaart worden gebracht aan de hand van een BPMN-schema. De verwachte tijdsbestek voor deze fase is vier weken.
\subsection{Fase 3: Interviewrondes}
Fase 3 is bedoeld om een gedetailleerder inzicht te verwerven in de tactieken van een managed service provider in relatie tot ERP-patchmanagement. Hiertoe zullen interviews plaatsvinden met bijvoorbeeld ERP-expertisehouders. Deze interviews proberen te achterhalen hoe de respondent naar het proces uit Fase 2 kijkt en waar volgens de respondent verbeteringen kunnen worden doorgevoerd. De geplande duur voor deze fase is drie weken.
\subsection{Fase 4: Analyse van Beschikbare Gereedschappen}
Op het grondige literatuuronderzoek volgt een evaluatie van voor handen zijnde gereedschappen en technologieën voor ERP-patchmanagement. In deze fase staan we stil bij hoe we het proces voor patchmanagement kunnen verbeteren met behulp van tooling met betrekking tot het procédé dat in Fase 2 is uitgebeeld. De verwachte periode voor deze fase is twee weken.
\subsection{Fase 5: Samenvatting en Adviezen}
In de slotfase worden de resultaten van het literatuuronderzoek, de analyses, gesprekken en het casusonderzoek samengebracht. Uit deze synthese worden conclusies getrokken en specifieke adviezen geformuleerd om de effectiviteit van ERP-patchmanagement in organisaties te verbeteren. Het tijdsbestek voor deze fase is twee weken. \