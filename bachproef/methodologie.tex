%%=============================================================================
%% Methodologie
%%=============================================================================

\chapter{\IfLanguageName{dutch}{Methodologie}{Methodology}}%
\label{ch:methodologie}

%% TODO: In dit hoofstuk geef je een korte toelichting over hoe je te werk bent
%% gegaan. Verdeel je onderzoek in grote fasen, en licht in elke fase toe wat
%% de doelstelling was, welke deliverables daar uit gekomen zijn, en welke
%% onderzoeksmethoden je daarbij toegepast hebt. Verantwoord waarom je
%% op deze manier te werk gegaan bent.
%% 
%% Voorbeelden van zulke fasen zijn: literatuurstudie, opstellen van een
%% requirements-analyse, opstellen long-list (bij vergelijkende studie),
%% selectie van geschikte tools (bij vergelijkende studie, "short-list"),
%% opzetten testopstelling/PoC, uitvoeren testen en verzamelen
%% van resultaten, analyse van resultaten, ...
%%
%% !!!!! LET OP !!!!!
%%
%% Het is uitdrukkelijk NIET de bedoeling dat je het grootste deel van de corpus
%% van je bachelorproef in dit hoofstuk verwerkt! Dit hoofdstuk is eerder een
%% kort overzicht van je plan van aanpak.
%%
%% Maak voor elke fase (behalve het literatuuronderzoek) een NIEUW HOOFDSTUK aan
%% en geef het een gepaste titel.

\subsection{Fase 1: Literatuurstudie}

In de eerste fase wordt een grondige literatuurstudie uitgevoerd om een diepgaand begrip te verkrijgen van de huidige stand van zaken in ERP-patchmanagement. Hierbij worden relevante bronnen geanalyseerd, waaronder wetenschappelijke artikelen en vakpublicaties. De focus ligt op het identificeren van bestaande patchmanagementpraktijken, uitdagingen, en succesfactoren. Deze fase zal naar verwachting twee weken in beslag nemen.

\subsection{Fase 2: Analyse van Beschikbare Tools en Technologieën}

Na de literatuurstudie volgt een analyse van beschikbare tools en technologieën voor ERP- patchmanagement. Dit omvat een evaluatie van bestaande softwareoplossingen, frameworks en technologische trends die relevant zijn voor het optimaliseren van patchprocessen. De resultaten zullen worden gebruikt om inzicht te krijgen in de technologische mogelijkheden en beperkingen. Deze fase wordt geschat op drie weken.

\subsection{Fase 3: Interviews}

Om een dieper inzicht te verkrijgen in de praktijken van groot- en middelgrote bedrijven op het gebied van ERP-patchmanagement, zullen interviews worden afgenomen bij relevante belanghebbenden, zoals IT-beheerders, security-experts en ERP-specialisten. De interviews zullen vragen bevatten om de huidige praktijken, uitdagingen en verwachtingen te achterhalen. Deze vragen zullen opgesteld worden met de opgedane kennis van voorafgaande fasen. De benodigde tijd hiervoor wordt ingeschat op drie weken.

\subsection{Fase 4: Case Studies}

Om de verzamelde informatie te valideren en specifieke situaties te begrijpen, zullen meerdere case studies worden uitgevoerd bij geselecteerde bedrijven. Hierbij worden de geïdentificeerde efficiëntie verhogende maatregelen geëvalueerd. De case studies bieden praktische inzichten en dienen als basis voor conclusies en aanbevelingen. Deze fase zal ongeveer drie weken in beslag nemen.

\subsection{Fase 5: Conclusie en Aanbevelingen}

In de laatste fase worden de bevindingen van de literatuurstudie, analyses, interviews en case studies samengebracht. Hieruit worden conclusies getrokken en worden concrete aanbevelingen geformuleerd voor het optimaliseren van ERP-patchmanagement in bedrijven. Deze fase wordt geschat op één week. \\

