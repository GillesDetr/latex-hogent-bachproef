%%=============================================================================
%% Samenvatting
%%=============================================================================

% TODO: De "abstract" of samenvatting is een kernachtige (~ 1 blz. voor een
% thesis) synthese van het document.
%
% Een goede abstract biedt een kernachtig antwoord op volgende vragen:
%
% 1. Waarover gaat de bachelorproef?
% 2. Waarom heb je er over geschreven?
% 3. Hoe heb je het onderzoek uitgevoerd?
% 4. Wat waren de resultaten? Wat blijkt uit je onderzoek?
% 5. Wat betekenen je resultaten? Wat is de relevantie voor het werkveld?
%
% Daarom bestaat een abstract uit volgende componenten:
%
% - inleiding + kaderen thema
% - probleemstelling
% - (centrale) onderzoeksvraag
% - onderzoeksdoelstelling
% - methodologie
% - resultaten (beperk tot de belangrijkste, relevant voor de onderzoeksvraag)
% - conclusies, aanbevelingen, beperkingen
%
% LET OP! Een samenvatting is GEEN voorwoord!

%%---------- Nederlandse samenvatting -----------------------------------------
%
% TODO: Als je je bachelorproef in het Engels schrijft, moet je eerst een
% Nederlandse samenvatting invoegen. Haal daarvoor onderstaande code uit
% commentaar.
% Wie zijn bachelorproef in het Nederlands schrijft, kan dit negeren, de inhoud
% wordt niet in het document ingevoegd.

\IfLanguageName{english}{%
\selectlanguage{dutch}
\chapter*{Samenvatting}
\selectlanguage{english}
}{}

%%---------- Samenvatting -----------------------------------------------------
% De samenvatting in de hoofdtaal van het document

\chapter*{\IfLanguageName{dutch}{Samenvatting}{Abstract}}


Deze bachelorproef onderzoekt patchmanagement bij ERP-systemen, de focus ligt op de optimalisatie ervan met behulp van tools en praktijken. De noodzaak voor dit onderzoek komt voort uit de groeiende complexiteit
van IT-infrastructuur en de toenemende afhankelijkheid van bedrijven met ERP-systemen. Ook is patchmanagement een belangrijk onderdeel van IT-beveiliging, het zorgt ervoor dat systemen beter beschermd tegen kwetsbaarheden.
Het onderzoek begint met een literatuurstudie om de huidige kennis rond deze bachelorproef te overlopen. Zo wordt het belang van tijdige communicatie met belanghebbenden benadrukt. Vervolgens wordt eerst het
grote voordeel van cloud ERP beheerd door de ERP-provider besproken, het geen nood meer hebben aan patching. Uitdagingen bij lokale machines worden belicht, zoals schaalbaarheid, kosteneffectiviteit en beveiligingsbeheer.
Daarnaast werd er een case study uitgevoerd bij een een managed service provider, dit gaf inzicht in de huidige situatie van patchmanagement. Hierdoor kon het proces in beeld worden gebracht door middel van een BPMN-schema en kon er ook gekeken worden naar verbeterpunten binnen het proces.

Uit dit BPMN-schema en case study bleek dat communicatie tussen de stakeholders het meeste tijd inneemt. Hieruit kan geconcludeerd worden dat een duidelijke communicatiestrategie hebben een belangrijk onderdeel is van het patchproces. Ook kunnen we zien in het proces dat er een deel van het patchen
geautomatiseerd kan worden, dit kan met behulp van patchmanagement tools hierop werd de focus gelegd op kernel patching. Er werd specifiek gekeken naar Avantra, een tool die het mogelijk maakt om kernel patches te uploaden en uit te voeren met de juiste
configuratie. Er werd gekeken naar de voordelen van deze tools, zoals het besparen van tijd en dus het verhogen van de efficiëntie. Maar ook de nadelen zoals de extra kosten, complexiteit en tijd die nodig zijn voor het implementeren van deze tools.
De toekomst van patchmanagement wordt ook overlopen, de opkomst van AI (artificiële intelligentie) zou zo in de toekomst bijvoorbeeld organisaties kunnen helpen om een prioriteit toe te kennen aan patches. Zo zou het proces nog efficiënter kunnen verlopen en zouden de patches sneller kunnen worden toegepast.

De resultaten van dit onderzoek tonen aan dat de implementatie procedures van belang is voor een succesvol patchmanagement binnen ERP-systemen, geautomatiseerde patchingtools zijn aan te raden voor bedrijven die
regelmatig kernel patching moeten doen. Voor SAP zijn er verschillende tools beschikbaar, zoals Avantra en SAP Landscape Management, die kunnen helpen bij het automatiseren van patchingprocessen. 
Voor niet SAP-ERP-systemen kan SolarWinds Patch Manager een geschikte patch automatisatietool zijn. De resultaten van dit onderzoek hebben enkele implicaties voor het werkveld, omdat ze aantonen hoe organisaties de betrouwbaarheid
van hun ERP-systemen kunnen blijven waarborgen door middel van sneller patches uit te voeren. De aanbevelingen benadrukken het belang van samenwerking en communicatie tussen IT-teams, cloudproviders en andere belanghebbenden, evenals de mogelijke noodzaak om te investeren in geautomatiseerde patchingtools voor patchmanagement indien er op regelmatige basis moet gepatched worden.