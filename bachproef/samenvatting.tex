%%=============================================================================
%% Samenvatting
%%=============================================================================

% TODO: De "abstract" of samenvatting is een kernachtige (~ 1 blz. voor een
% thesis) synthese van het document.
%
% Een goede abstract biedt een kernachtig antwoord op volgende vragen:
%
% 1. Waarover gaat de bachelorproef?
% 2. Waarom heb je er over geschreven?
% 3. Hoe heb je het onderzoek uitgevoerd?
% 4. Wat waren de resultaten? Wat blijkt uit je onderzoek?
% 5. Wat betekenen je resultaten? Wat is de relevantie voor het werkveld?
%
% Daarom bestaat een abstract uit volgende componenten:
%
% - inleiding + kaderen thema
% - probleemstelling
% - (centrale) onderzoeksvraag
% - onderzoeksdoelstelling
% - methodologie
% - resultaten (beperk tot de belangrijkste, relevant voor de onderzoeksvraag)
% - conclusies, aanbevelingen, beperkingen
%
% LET OP! Een samenvatting is GEEN voorwoord!

%%---------- Nederlandse samenvatting -----------------------------------------
%
% TODO: Als je je bachelorproef in het Engels schrijft, moet je eerst een
% Nederlandse samenvatting invoegen. Haal daarvoor onderstaande code uit
% commentaar.
% Wie zijn bachelorproef in het Nederlands schrijft, kan dit negeren, de inhoud
% wordt niet in het document ingevoegd.

\IfLanguageName{english}{%
\selectlanguage{dutch}
\chapter*{Samenvatting}
\selectlanguage{english}
}{}

%%---------- Samenvatting -----------------------------------------------------
% De samenvatting in de hoofdtaal van het document

\chapter*{\IfLanguageName{dutch}{Samenvatting}{Abstract}}

Deze bachelorproef onderzoekt patchmanagement in cloudgebaseerde ERP-systemen, met een focus op communicatiestrategieën voor belanghebbenden en de toekomst van patchmanagement met behulp van verschillende tools. De noodzaak voor dit onderzoek komt voort uit de groeiende complexiteit van IT-infrastructuur en de toenemende afhankelijkheid van bedrijven van cloudgebaseerde ERP-systemen.

Het onderzoek start met het benadrukken van het belang van heldere en tijdige communicatie met belanghebbenden over geplande patchactiviteiten, evenals het bieden van ondersteuning aan gebruikers. Vervolgens wordt de overgang naar cloudgebaseerde ERP-systemen besproken, waarbij de voordelen zoals het het geen nood meer hebben aan patching. Ook uitdagingen worden belicht, zoals schaalbaarheid, kosteneffectiviteit en beveiligingsbeheer.

De resultaten van dit onderzoek tonen aan dat de implementatie van geautomatiseerde patchingtools en procedures van vitaal belang is voor een succesvol patchmanagement binnen ERP-systemen. Voor SAP (de meest gebruikte ERP-software) zijn er verschillende tools beschikbaar, zoals Avantra en Sap Landscape Management, die kunnen helpen bij het automatiseren van patchingprocessen. 

Voor niet SAP-ERP-systemen kan SolarWinds Patch Manager een goede patchautomatisatietool zijn.

Daarnaast is het van cruciaal belang om een duidelijke communicatiestrategie te hebben voor belanghebbenden, zodat ze op de hoogte zijn van geplande patchactiviteiten en eventuele impact op hun werkzaamheden.

Deze bevindingen hebben belangrijke implicaties voor het werkveld, omdat ze aantonen hoe organisaties de betrouwbaarheid van hun cloudgebaseerde ERP-systemen kunnen waarborgen. De aanbevelingen benadrukken het belang van samenwerking en communicatie tussen IT-teams, cloudproviders en andere belanghebbenden, evenals de noodzaak om te investeren in geautomatiseerde patchingtools voor patchmanagement.