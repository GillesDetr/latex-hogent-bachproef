\chapter{\IfLanguageName{dutch}{Stand van zaken}{State of the art}}
\label{ch:stand-van-zaken}

% Tip: Begin elk hoofdstuk met een inleidende paragraaf die beschrijft hoe
% dit hoofdstuk past binnen het geheel van de bachelorproef.
% Geef in het bijzonder aan wat de link is met het vorige en volgende hoofdstuk.

In dit hoofdstuk wordt de stand van zaken met betrekking tot ERP-systemen en patchmanagement behandeld. De focus ligt op het belang van patchmanagement binnen ERP-systemen, uitdagingen en strategieën die gepaard gaan met het beheren van patches in deze omgeving.

\section{Inleiding tot ERP-systemen en patchmanagement}
Enterprise Resource Planning (ERP) systemen zijn niet weg te denken van moderne bedrijfsvoering. Deze software speelt een essentiële rol bij het stroomlijnen en automatiseren van verschillende bedrijfsprocessen, variërend van financiën en voorraadbeheer tot human resources en klantrelatiebeheer.

Door het centraliseren van gegevens en processen helpen ERP-systemen organisaties om efficiënter te werken, betere beslissingen te nemen en hun concurrentiepositie te versterken. Met

de toenemende afhankelijkheid van ERP-systemen komt ook de groeiende dreiging van cyberaanvallen. Als centrale hubs voor gevoelige bedrijfsgegevens, worden ERP-systemen vaak het doelwit van kwaadwillende actoren die uit zijn op gegevensdiefstal, verstoring van 

bedrijfsprocessen of financieel gewin. In deze context wordt het belang van patchmanagement binnen ERP-systemen steeds groter, vooral om de veiligheid en stabiliteit van de systemen te waarborgen \autocite{Pearson2024}.

Het implementeren van een ERP-systeem is een cruciale stap voor organisaties van een bepaalde omvang, meestal ergens rond een paar honderd werknemers. Zelfs sommige kleinere organisaties met complexe activiteiten zullen de noodzaak van ERP vinden. De vraag 

naar ERP heeft een aanzienlijke markt gecreëerd. Volgens \textcite{Madh2024} was de wereldwijde ERP-softwaremarkt in 2023 \$49,80 miljard en wordt verwacht dat deze tegen 2030 zal stijgen naar \$140,14 miljard. Drie grote namen - 

Microsoft, Oracle en SAP - domineren de markt, maar verschillende kleinere spelers bieden ook ERP-producten aan die op vele manieren concurrerend zijn met die van de marktleiders \autocite{Pratt2023}.

Patchmanagement verwijst naar het proces van het identificeren, beoordelen, testen en implementeren van softwarepatches om bekende kwetsbaarheden in systemen te verhelpen. Binnen het domein van ERP-systemen, is patchmanagement van cruciaal belang

om de veiligheid en stabiliteit van de systemen te waarborgen. Door regelmatig patches toe te passen, kunnen organisaties potentiële beveiligingsrisico's minimaliseren en zichzelf beschermen tegen externe bedreigingen \autocite{Buenning2024}.

\section{Managed service providers}

Een managed service provider (MSP) is een extern bedrijf dat de IT-infrastructuur en eindgebruikerssystemen van een klant op afstand beheert. Bedrijven huren MSP's in om een pakket aan dagelijkse beheerdiensten uit te voeren (zoals patchmanagement).

MSP's beheren vaak dagelijkse diensten zodat klantorganisaties zich kunnen richten op het verbeteren van hun eigen diensten, zonder zich zorgen te hoeven maken over langdurige systeemuitval of onderbrekingen van de dienstverlening.

In de context van ERP-patchmanagement spelen MSP's een belangerijke rol voor hun klanten. ERP-systemen zijn complexe worden geintegreerd met veel bedrijfsprocessen, het up-to-date houden van deze systemen is dus essentieel voor de beveilliging van hun klanten. \autocite{Gillis2021}

\section{Soorten patches}
Volgens \textcite{Buenning2024} zijn er drie types patches, namelijk beveiligingspatches, bugfixes en feature patches. Beveiligingspatches, die het aanpakken van nieuw ontdekte beveiligingskwetsbaarheden in het systeem inhouden, zijn een cruciaal aspect van het handhaven van de integriteit en beveiliging van elke softwareomgeving. Naast beveiligingspatches spelen bugfixes ook een essentiële rol bij het zorgen 

voor de soepele werking van softwaresystemen door problemen of fouten aan te pakken die zich kunnen voordoen in de functionaliteit van het systeem. Daarnaast zijn feature patches essentieel voor het verbeteren van de algehele prestaties van het systeem. Deze patches zijn bedoeld om de resource-eisen te 

verminderen, de snelheid van toepassingen te verbeteren en nieuwe functionaliteiten te introduceren om de gebruikerservaring en efficiëntie te optimaliseren. \\

Binnen ERP-systemen moet de kern van het besturingssysteem, namelijk de kernel af en toe gepatched worden, dit is dan met met een van de drie types patches. In ERP-systemen speelt de database ook een zeer belangerijke rol, hiervoor zij er ook patches nodig. Deze patches zorgen ervoor dat de database optimaal blijft werken en dat de data veilig is.

Het is van cruciaal belang voor organisaties om regelmatig de risico's te beoordelen en dus het implementeren van patches om de risico's te beperken en de soepele werking van hun systemen te garanderen.

Deze concepten worden weerspiegeld in het onderzoek van \textcite{Wrobel2023}, waarin het belang van patchmanagementstrategieën wordt benadrukt bij het aanpakken van beveiligingskwetsbaarheden, bugs en prestatieproblemen binnen softwareomgevingen.

\section{Patchmanagement vs Vulnerability management}
Patchmanagement is specifiek gericht op het toepassen van patches om bekende kwetsbaarheden aan te pakken. Vulnerability management bevat breder scala aan activiteiten. Deze activiteiten omvatten configuratiebeheer, beveiligingsbewustzijnstraining en penetratietesten, allemaal gericht op het verbeteren van de algehele beveiliging en het beheersen van risico’s \autocite{Danby2023}


\section{Trends en ontwikkelingen in  ERP en patchmanagement}

De eerste belangrijke trend is de nadruk op proactieve beveiligingsmaatregelen.

In plaats van te wachten tot zich kwetsbaarheden voordoen, streven organisaties ernaar om potentiële beveiligingsrisico's voor te zijn door regelmatig patches toe te passen en beveiligingsupdates te implementeren.

Nog een belangrijke trend is dat ontwikkelingen binnen ERP bedrijven hebben geleid tot een groeiende afhankelijkheid van cloudservices voor patchmanagement.

Steeds meer organisaties verplaatsen hun ERP-systemen naar de cloud om te profiteren van schaalbaarheid, flexibiliteit en lagere operationele kosten.

Deze verschuiving naar de cloud heeft ook invloed op patchmanagement, aangezien organisaties nu op zoek zijn naar cloudgebaseerde oplossingen voor het beheren en implementeren van patches over hun gedistribueerde netwerken en apparaten. \autocite{Kannan2023}

\section{Uitdagingen bij patching}
Organisaties worden geconfronteerd met verschillende uitdagingen bij het beheren van patches. Een van de meest

voorkomende problemen is de traagheid waarmee updates worden uitgevoerd. Dit kan te wijten zijn aan verschillende

factoren, waaronder het gebrek aan middelen of prioriteitstoewijzing binnen de IT-afdeling. Als gevolg hiervan blijven

systemen mogelijk kwetsbaar voor bekende beveiligingsrisico's, wat de algehele veiligheid van het bedrijf in gevaar kan brengen. \autocite{AppMaster2023}

Ook werken veel organisaties met meerdere systemen en toepassingen.

Dit betekent dat IT-professionals verschillende besturingssystemen moeten kunnen patchen, zoals Linux waar erp systemen op draaien, evenals verschillende applicaties van derden.

Het patchen van al deze systemen kan dus een hele klus zijn. Het patchen van systemen wordt vaak als tijdrovend en complex ervaren door IT- en beveiligingsprofessionals.

Volgens \textcite{ivanti2021} vindt maar liefst 71\% van hen patchen te complex en tijdrovend. Het handmatig patchen van elk apparaat op een netwerk is niet alleen frustrerend en langzaam, maar ook buitengewoon inefficiënt voor grotere organisaties.

Dit leidt tot grote kosten voor bedrijven die veel moeten patchen, waardoor sommigen beroep doen op landen waar de kostprijs per werkuur veel lager ligt dan in België. \autocite{Munck2024}

Aangezien ERP-systemen vaak complexe architecturen hebben en geïntegreerd zijn met een breed scala aan andere systemen en applicaties, kunnen patches soms onverwachte problemen veroorzaken. Dit kan leiden tot verstoring van bedrijfsactiviteiten 

en downtime, wat op zijn beurt kan leiden tot financiële verliezen en reputatieschade voor het bedrijf.


\section{De keuze tussen handmatig en automatisch patchen}

Binnen organisaties die SAP ERP-systemen gebruiken, staan IT-teams vaak voor de keuze tussen handmatig of automatisch patchen. Beide benaderingen hebben hun eigen voor- en nadelen, en het is cruciaal dat organisaties de juiste keuze maken op basis van hun specifieke behoeften en omgeving.

Handmatig patchen geeft IT-teams veel controle en flexibiliteit bij het beheren van patches in ERP systemen. Met deze aanpak kan elk patch-, klant- of server-combinatie afzonderlijk worden behandeld. Hierdoor kunnen alleen de noodzakelijke patches worden geïnstalleerd, precies wanneer dat nodig is, wat het risico op onbedoelde gevolgen vermindert. Compatibiliteitsproblemen kunnen worden verminderd en de algehele stabiliteit van het ERP-systeem wordt gewaarborgd \autocite{Hooper2018}.

Het nadeel van handmatig patchen is echter dat het tijdrovend is en veel menselijke middelen vereist. IT-teams moeten voortdurend patchbulletins volgen, patches testen en implementeren, wat kan leiden tot vertragingen bij het toepassen van cruciale updates. Bovendien vergroot de menselijke factor het risico op fouten, waardoor organisaties kwetsbaar kunnen zijn voor beveiligingsrisico’s door gemiste patches of onjuiste implementaties.

Automatisch patchen daarentegen biedt een hoog niveau van efficiëntie en consistentie bij het beheren van patches in ERP systemen. Met automatische patchingtools kunnen patches toegepast worden volgens een vooraf ingesteld schema. Organisaties moeten

ervoor zorgen dat ze de juiste patchingtools hebben geïmplementeerd die zijn afgestemd op hun specifieke behoeften en omgeving. Er bestaat ook het risico dat automatische patchingtools incompatibele patches toepassen, wat kan leiden tot systeemstoringen of conflicten met andere softwarecomponenten.

Volgens \textcite{Tozzi2017} is het beste om een evenwichtige updatestrategie te hanteren. Deze strategie maakt gebruik van automatische updates waar ze nuttig zijn, maar vermijdt ze in situaties waarin ze te veel risico met zich meebrengen of niet toereikend zijn. Updatestrategieën moeten natuurlijk worden afgestemd op de specifieke behoeften van de organisatie.

Over het algemeen zou een goed ontworpen updatestrategie automatische updates richten op kritieke beveiligingslekken, updates voor het besturingssysteem zelf, en systemen die gemakkelijk kunnen worden teruggerold. Aan de andere kant zijn updates voor firmware, randapparatuur, netwerkschakelaars en andere soorten software die niet betrouwbaar kunnen worden geautomatiseerd, evenals niet-kritieke updates en updates voor systemen die zeer beschikbaar moeten zijn, beter geschikt voor handmatige patching.

Deze aanpak helpt om een balans te vinden tussen up-to-date en veilig zijn aan de ene kant, en de stabiliteit van uw software aan de andere kant.

\section{Wanneer patchen?}
Een kosten baten analyse wordt vaak gemaakt om te bepalen wanneer er gepatcht moet worden. Hier worden de kosten van het patchen afgewogen tegen de kosten van een eventuele cyberaanvallen.

In de figuur van \textcite{Posey2024} zien we hoe de kosten baten analyse eruit kan zien.

\begin{figure}[htbp]
    \centering
    \includegraphics[width=\textwidth]{techtarget.jpg}
    \caption{Kosten baten analyse patching \autocite{Posey2024}}
    \label{fig:kostenbaten}
\end{figure}


\section{Invloed van menselijke, technologische en organisatorische factoren op patchmanagement}

Patchmanagement binnen ERP-systemen wordt sterk beïnvloed door een combinatie van menselijke, technologische en organisatorische factoren. Een 
diepgaande analyse van deze elementen is essentieel om een effectieve patchmanagementstrategie te ontwikkelen en te implementeren. Menselijke 
factoren spelen een cruciale rol in het patchmanagementproces. Het bewustzijn van medewerkers over de noodzaak van patching en het belang van cybersecurity
is van vitaal belang. Zonder voldoende bewustwording kunnen patches over het hoofd worden gezien of niet tijdig worden toegepast, waardoor de veiligheid van het ERP-systeem in gevaar komt. Daarnaast 
is training van medewerkers essentieel om ervoor te zorgen dat ze de juiste procedures volgen bij het beoordelen, testen en implementeren van patches. 
Een goed opgeleid team kan de efficiëntie van het patchmanagementproces aanzienlijk verbeteren en de kans op menselijke fouten minimaliseren. Technologische 
factoren omvatten de beschikbaarheid en kwaliteit van patchingtools en -technologieën binnen ERP-systemen. Geavanceerde

patchingtools kunnen het patchmanagementproces automatiseren en vereenvoudigen, waardoor de werklast voor IT-teams wordt verminderd en de kans op fouten wordt verkleind. 

Het gebruik van geautomatiseerde tools kan ook zorgen voor een snellere detectie en reactie op beveiligingsdreigingen,

wat cruciaal is voor het beschermen van het ERP-systeem tegen potentiële aanvallen. Organisatorische factoren, zoals 
de samenwerking tussen verschillende afdelingen en de implementatie van duidelijke patchmanagementprocedures, zijn ook van groot belang. Een goed
 gestructureerd patchmanagementbeleid, ondersteund door heldere communicatiekanalen en duidelijke verantwoordelijkheden, kan de efficiëntie van het patchmanagementproces aanzienlijk verbeteren. Door nauw
  samen te werken met verschillende belanghebbenden, waaronderIT-teams, beveiligingsexperts en operationele afdelingen, kunnen organisaties ervoor zorgen dat patching prioriteit krijgt en effectief wordt uitgevoerd.


In het specifieke geval van SAP ERP is het belangrijk om te benadrukken dat patchmanagement niet alleen een technische kwestie is, maar ook een strategisch aspect van IT-beveiliging.

Het succesvol beheren van patches vereist daarom een holistische benadering waarbij rekening wordt gehouden met zowel menselijke, technologische als organisatorische factoren.

Door bewustwording te vergroten, medewerkers te trainen, geavanceerde tools te gebruiken en effectieve samenwerking te bevorderen, 

kunnen organisaties hun SAP ERP-systemen adequaat beschermen tegen potentiële bedreigingen en de operationele continuïteit waarborgen \autocite{Graffeo2018}.

\section{Strategieën voor minimalisering van de impact op de continuïteit}
Effectieve patching vereist zorgvuldige planning om de impact te minimaliseren. Het is raadzaam patchactiviteiten te plannen tijdens niet-kritieke operationele periodes.

Daarnaast kan een gefaseerde implementatie, beginnend met minder kritieke systemen, de operationele continuïteit waarborgen. Het implementeren van fallbackstrategieën is essentieel voor 

een snelle en efficiënte herstelprocedure in geval van onvoorziene problemen. Patching kan falen

door verschillende redenen, variërend van incompatibele hardware tot conflicten met andere patches, of een patch die goed installeert maar iets anders kapotmaakt \autocite{Shein2022}.

Het implementeren van patches binnen SAP ERP-systemen kan een aanzienlijke invloed hebben op de operationele continuïteit van een organisatie. Om deze impact te minimaliseren,

zijn er verschillende strategieën die organisaties kunnen toepassen tijdens het patchmanagementproces. Ten eerste is zorgvuldige planning essentieel, patchactiviteiten kan best gebeuren

tijdens niet-kritieke operationele periodes, zoals buiten de reguliere kantooruren of tijdens perioden van verminderde bedrijfsactiviteit. Door patches op strategische momenten toe te passen,

kan de verstoring van lopende processen tot een minimum worden beperkt. Daarnaast is een gefaseerde implementatie van patches een effectieve strategie om de impact op de continuïteit te minimaliseren. 

In plaats van alle systemen tegelijkertijd te patchen, kunnen organisaties ervoor kiezen om patches geleidelijk aan te brengen, te beginnen met

minder kritieke systemen. Op deze manier kunnen eventuele problemen of compatibiliteitsproblemen

worden geïdentificeerd en opgelost voordat de patches op cruciale systemen worden toegepast. Het opzetten van fallbackstrategieën is ook essentieel voor een succesvol patchmanagementproces.

Ondanks zorgvuldige planning en gefaseerde implementatie kunnen er onvoorziene problemen optreden tijdens het patchen, zoals systeemstoringen of conflicten tussen patches. Door fallbackstrategieën te definiëren en vooraf te plannen,

kunnen organisaties snel en efficiënt reageren op dergelijke situaties en de operationele continuïteit waarborgen. Dit kan bijvoorbeeld het creëren van back-ups van cruciale systemen omvatten, 

zodat ze snel kunnen worden hersteld in geval van problemen tijdens het patchen. In het kort benadrukken zorgvuldige planning, gefaseerde

implementatie en fallbackstrategieën het belang van een proactieve en strategische benadering van patchmanagement. Door deze strategieën

toe te passen, kunnen organisaties de impact van patchactiviteiten op de operationele continuïteit minimaliseren en tegelijkertijd de veiligheid en stabiliteit van hun SAP ERP-systemen waarborgen.

\section{Communicatiestrategieën voor belanghebbenden}
Heldere en tijdige communicatie met belanghebbenden over geplande patchactiviteiten is van vitaal belang. Het is essentieel om begrijpelijke informatie te verstrekken over de redenen achter het patchen,

mogelijke impact en benodigde voorzorgsmaatregelen \autocite{Toren2019}. Daarnaast zijn het opzetten van communicatiekanalen voor feedback en het bieden van ondersteuning aan gebruikers eveneens cruciaal voor een succesvolle implementatie.


\section{Patchmanagement in cloudgebaseerde ERP-systemen}
Volgens \textcite{Forbes2021} kan de cloud een meer gestroomlijnde benadering bieden, terwijl tegelijkertijd geprofiteerd wordt van beveiligings best practices en updates van cloud- en softwareleveranciers. Volgens \textcite{Ruiter2024} bieden

cloudgebaseerde ERP-systemen verschillende voordelen ten opzichte van traditionele on-premises oplossingen. Ze kunnen gemakkelijk meeschalen met de groeiende behoeften van een organisatie, 

vereenvoudigen het implementatieproces van beveiligingspatches en bieden realtime zichtbaarheid in de patchstatus van alle apparaten, ongeacht hun locatie. De initiële kosten 

zijn lager bij de cloudinfrastructuur, en deze kosten zijn ook meer voorspelbaar, terwijl bij een on-premise ERP-systeem de initiële kosten veel hoger kunnen zijn, 

maar de totale kosten na verloop van tijd lager kunnen uitvallen, afhankelijk van de vereisten. Een ander voordeel van de cloudprovider is dat

de verantwoordelijkheid voor data beveiliging nu ook bij hen ligt.De interesse in ERP

overstijgt verticale industrieën, met fabrikanten, dienstverleners, non-profitorganisaties en overheidsentiteiten die allemaal behoefte hebben aan de mogelijkheden ervan om efficiënt en effectief te kunnen

functioneren. In de afgelopen jaren zijn velen overgestapt van decennia-oude on-premises ERP-systemen naar nieuwe cloudgebaseerde ERPs. Anderen die nooit ERP hadden, gaan rechtstreeks naar de cloudoptie, vooral 

SaaS ERP. We zien ook dat de markt dit reflecteert en dat nu minder dan een derde van de bedrijfsapplicaties naar verwachting in 2022 nog gehost zal worden op traditionele servers, terwijl de rest zal vertrouwen op cloud

computing-oplossingen. Bedrijven over de hele wereld vertrouwen nu steeds meer op cloud computing om hun bedrijf te runnen. Het groeit zo snel in populariteit dat minder dan een derde van de bedrijfsapplicaties nog verwacht wordt 

te worden gehost op traditionele servers tegen 2022, en de rest zal vertrouwen op cloud computing-oplossingen \autocite{Pimentel2017}.

Patchmanagement in cloudgebaseerde ERP-systemen is dus een steeds belangrijker wordend aspect van IT-beheer voor moderne organisaties. Met de opkomst van cloud computing hebben veel bedrijven ervoor gekozen

om hun ERP-systemen naar de cloud te verplaatsen om te profiteren van de schaalbaarheid, flexibiliteit en kosteneffectiviteit die de cloud biedt. 

Echter, het beheren van patches in een cloudomgeving brengt unieke uitdagingen met zich mee. Om succesvol patchmanagement

in cloudgebaseerde ERP-systemen te implementeren, moeten organisaties investeren in geautomatiseerde patchingtools, 

beleidsregels en procedures voor patchmanagement, en continue monitoring en rapportage van patch statussen. Daarnaast is samenwerking en communicatie tussen IT-teams, 

cloudproviders en andere belanghebbenden van vitaal belang om ervoor te zorgen dat patches tijdig worden geïmplementeerd zonder de 

operationele continuïteit te verstoren. Met de juiste aanpak kunnen organisaties de beveiliging en betrouwbaarheid van hun cloudgebaseerde ERP-systemen waarborgen in een snel evoluerend technologisch landschap.


\section{De toekomst van patchmanagement}
De toekomst van AI in patchmanagement ziet er veelbelovend uit, vooral met de groeiende uitdagingen waar organisaties mee te maken hebben bij

het beheren van patches. Elke week worden er steeds meer patches uitgebracht, wat het moeilijk maakt om te bepalen welke patches moeten worden toegepast, hoe snel dat moet gebeuren en waar. Dit zorgt voor een aanzienlijke werklast voor patchmanagementprogramma's, die nog verder toeneemt naarmate

organisaties meer eindpunten en diverse softwarebronnen toevoegen. Steeds meer bedrijven wenden zich tot AI en machine learning

om deze uitdagingen aan te pakken. Deze technologieën kunnen helpen bij het detecteren, prioriteren en snel toepassen van patches wanneer dat nodig is. Dit leidt tot een efficiëntere werking, waardoor de algehele beveiliging wordt verbeterd door kwetsbaarheden sneller te 

identificeren en de kans te verkleinen dat deze worden uitgebuit in aanvallen.

AI-algoritmen begrijpen de complexe relaties tussen verschillende variabelen en kunnen een patchschema aanbevelen dat is afgestemd op de specifieke behoeften van een organisatie. Daarnaast kunnen AI-tools compatibiliteitsrisico's verminderen door slimme implementatietests uit te voeren en de belasting van IT-resources te verlagen. Met AI kunnen bedrijven ook
 endpoint- en gebruikersprofielen evalueren, zodat alleen relevante patches op het juiste moment worden toegepast, met minimale impact op gebruikers en bedrijfsactiviteiten. Hoewel de toepassing van AI in patchmanagement veel potentieel heeft, zijn er ook uitdagingen en risico's. Het gebruik van AI in patchmanagement is nog nieuw en organisaties moeten een steile leercurve doorlopen om deze technologie effectief te implementeren. Bovendien zijn er ethische 

overwegingen bij het gebruik van AI voor autonome beslissingen over patchprioritering en -implementatie. AI is niet perfect en de voorspellingen zijn niet altijd nauwkeurig, wat kan leiden tot fouten bij het beoordelen van de impact van patches​ \autocite{OFlaherty2023}

\section{Patchmanagement best practices}
Om patchmanagement effectief te kunnen implementeren binnen een organisatie zijn bepaalde stappen noodzakelijk. Een uitgebreide beoordeling van alle 
ERP-systemen of -programma's zou voor een bedrijf zeer nuttig blijken. Dit maakt de identificatie van ontbrekende patches mogelijk op basis van hun 
ernstniveaus en zorgt ervoor dat prioriteiten kunnen worden gesteld. Vervolgens wordt de planning van patch-implementaties cruciaal: deze moeten worden 
georganiseerd zonder nadelige gevolgen voor de productiviteit van werknemers, doorgaans door afzonderlijke implementatieschema's te gebruiken voor verschillende 
afdelingen of systemen die elkaar niet hinderen. Een patchbeheerprogramma bestaat doorgaans uit tools die automatisch patches kunnen implementeren op basis 
van de beschikbaarheid van de gebruiker en de uptime van het systeem. Naast de strategieën die zijn ontwikkeld om alle programma's te patchen, zijn het 
testen van patches na de implementatie en het garanderen van een roll-back-mogelijkheid voor patches voor het geval ze problemen veroorzaken ook noodzakelijke 
stappen voor een effectief patchbeheerproces. Het is van cruciaal belang om een ​​centraal punt voor de patchbeheeroplossing te selecteren, de nadruk te leggen 
op de patches die moeten worden geïmplementeerd en op het patchproces. \autocite{ManageEngine2024}


