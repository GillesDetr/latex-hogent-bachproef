\chapter{\IfLanguageName{dutch}{Stand van zaken}{State of the art}}%
\label{ch:stand-van-zaken}

% Tip: Begin elk hoofdstuk met een paragraaf inleiding die beschrijft hoe
% dit hoofdstuk past binnen het geheel van de bachelorproef.

 Geef in het
% bijzonder aan wat de link is met het vorige en volgende hoofdstuk.



% Pas na deze inleidende paragraaf komt de eerste sectiehoofding.



\section{Inleiding tot ERP-systemen en patchmanagement}
Enterprise Resource Planning (ERP) systemen vormen de ruggengraat van moderne bedrijfsvoering.
 Deze geïntegreerde softwaretoepassingen spelen een essentiële rol bij het stroomlijnen en automatiseren van verschillende bedrijfsprocessen, variërend van financiën en voorraadbeheer tot human resources en klantrelatiebeheer.
 Door het centraliseren van gegevens en processen helpen ERP-systemen organisaties om efficiënter te werken, betere beslissingen te nemen en hun concurrentiepositie te versterken.


Het belang van ERP-systemen voor bedrijven kan niet worden onderschat.

 \textcite{StatistiekVlaanderen2022} toont aan dat een aanzienlijk deel van de bedrijven, met name in de groot- en middelgrote sector, vertrouwt op ERP-software om hun dagelijkse activiteiten te stroomlijnen en te beheren.

 Deze systemen bieden een gestandaardiseerde aanpak voor gegevensverwerking en rapportage, wat resulteert in verbeterde operationele efficiëntie en nauwkeurigheid.

Echter, met de toenemende afhankelijkheid van ERP-systemen komt ook de groeiende dreiging van cyberaanvallen.
 Als centrale hubs voor gevoelige bedrijfsgegevens, worden ERP-systemen vaak het doelwit van kwaadwillende actoren die uit zijn op gegevensdiefstal, verstoring van bedrijfsprocessen of financieel gewin.
 In deze context wordt het belang van patchmanagement binnen ERP-systemen steeds groter.
 Ook met de komst van AI zal het aantal cybersecurity risico’s worden vergroot, doordat hackers veel minder kennis nodig zullen hebben om een systeem te hacken.
  \autocite{Pearson2024}
Het implementeren van een ERP-systeem is een cruciale stap voor organisaties van een bepaalde omvang, meestal ergens rond een paar honderd werknemers.
 Zelfs sommige kleinere organisaties met complexe activiteiten zullen de noodzaak van ERP vinden.

De vraag naar ERP heeft een aanzienlijke markt gecreëerd.
 Volgens \textcite{Madh2024} was de wereldwijde ERP-softwaremarkt op \$49,80 miljard in 2023 en schatten dat deze tegen 2030 zal stijgen naar \$140,14 miljard.

Enterprise-leiders hebben tientallen keuzes als het gaat om ERP.
 Drie grote namen - Microsoft, Oracle en SAP - domineren de markt, maar verschillende kleinere spelers bieden ERP-producten aan die op vele manieren concurrerend zijn met die van de marktleiders.
 \autocite{Pratt2023}
Patchmanagement verwijst naar het proces van het identificeren, beoordelen, testen en implementeren van softwarepatches om bekende kwetsbaarheden in systemen te verhelpen.
 
Binnen het domein van ERP-systemen, met name binnen SAP ERP, is patchmanagement van cruciaal belang om de veiligheid en stabiliteit van de systemen te waarborgen.

 Door regelmatig patches toe te passen, kunnen organisaties potentiële beveiligingsrisico's minimaliseren en zichzelf beschermen tegen externe bedreigingen.


Het concept van patchen heeft zijn wortels in de begindagen van softwareontwikkeling en IT-systemen.

 In de vroege jaren van de computerindustrie werden bugs en kwetsbaarheden ontdekt in softwareprogramma's nadat ze al in gebruik waren genomen.

 Om deze problemen op te lossen, ontwikkelden softwarebedrijven patches die specifiek waren ontworpen om problemen in de software op te lossen zonder de noodzaak van een volledige heruitgave van het programma.

 \autocite{Buenning2024}
De behoefte aan gestandaardiseerde patchmanagementpraktijken werd nog urgenter naarmate de complexiteit van softwaretoepassingen toenam en het aantal bedreigingen van buitenaf groeide.

 In de loop van de tijd werden patching-processen steeds formeler en geïnstitutionaliseerd, met de ontwikkeling van specifieke protocollen en richtlijnen voor het beheer van patches.

 
De groeiende dreiging van cyberaanvallen en malware-infecties heeft de noodzaak van effectief patchmanagement nog verder benadrukt.

 Hackers maken vaak gebruik van bekende kwetsbaarheden in software om systemen binnen te dringen en schade aan te richten.

 Als reactie hierop zijn patchmanagementpraktijken geëvolueerd om zich niet alleen te richten op het oplossen van bugs, maar ook op het snel aanpakken van beveiligingslekken om de IT-infrastructuur van organisaties te beschermen.



\section{Soorten patches}
Beveiligingspatches, die het aanpakken van nieuw ontdekte beveiligingskwetsbaarheden in het systeem inhouden, zijn een cruciaal aspect van het handhaven van de integriteit en beveiliging van elke softwareomgeving.

 Naast beveiligingspatches spelen bugfixes een essentiële rol bij het zorgen voor de soepele werking van softwaresystemen door problemen of fouten aan te pakken die zich kunnen voordoen in de functionaliteit van het systeem.

 Daarnaast zijn prestatiepatches essentieel voor het verbeteren van de algehele prestaties van het systeem.

 Deze patches zijn bedoeld om de resource-eisen te verminderen, de snelheid van toepassingen te verbeteren en nieuwe functionaliteiten te introduceren om de gebruikerservaring en efficiëntie te optimaliseren.


Deze verschillende soorten patches dragen gezamenlijk bij aan de stabiliteit, beveiliging en efficiëntie van softwaresystemen.

 Het is van cruciaal belang voor organisaties om regelmatig deze patches te beoordelen en te implementeren om risico's te beperken en de soepele werking van hun systemen te waarborgen.


Deze concepten worden weerspiegeld in het onderzoek van \textcite{Wrobel2023}, waarin het belang van patchmanagementstrategieën wordt benadrukt bij het aanpakken van beveiligingskwetsbaarheden, bugs en prestatieproblemen binnen softwareomgevingen.


\section{Patchmanagement vs Vunerability management}
Beveiligingspatches, die het aanpakken van nieuw ontdekte beveiligingskwetsbaarheden in het systeem inhouden, zijn een cruciaal aspect van het handhaven van de integriteit en beveiliging van elke softwareomgeving.

 Naast beveiligingspatches spelen bugfixes een essentiële rol bij het zorgen voor de soepele werking van softwaresystemen door problemen of fouten aan te pakken die zich kunnen voordoen in de functionaliteit van het systeem.

 Daarnaast zijn prestatiepatches essentieel voor het verbeteren van de algehele prestaties van het systeem.

 Deze patches zijn bedoeld om de resource-eisen te verminderen, de snelheid van toepassingen te verbeteren en nieuwe functionaliteiten te introduceren om de gebruikerservaring en efficiëntie te optimaliseren.


Deze verschillende soorten patches dragen gezamenlijk bij aan de stabiliteit, beveiliging en efficiëntie van softwaresystemen.

 Het is van cruciaal belang voor organisaties om regelmatig deze patches te beoordelen en te implementeren om risico's te beperken en de soepele werking van hun systemen te waarborgen.


Deze concepten worden weerspiegeld in het onderzoek van \textcite{Wrobel2023}, waarin het belang van patchmanagementstrategieën wordt benadrukt bij het aanpakken van beveiligingskwetsbaarheden, bugs en prestatieproblemen binnen softwareomgevingen.



Patchmanagement vs Vunerability management
Patchmanagement richt zich op het identificeren, implementeren en beheren van software-updates (patches) om bekende beveiligingslekken in systemen en applicaties te verhelpen.

 Het hoofddoel is om de beveiliging te versterken door kwetsbaarheden te dichten en het risico op succesvolle aanvallen te verminderen.

 Aan de andere kant omvat vulnerability management het identificeren, evalueren en beheren van kwetsbaarheden in systemen en netwerken, zowel bekende als onbekende, met als doel een proactieve benadering van beveiliging te bevorderen door kwetsbaarheden te verminderen of te elimineren.

 Terwijl patchmanagement zich specifiek richt op het toepassen van patches om bekende kwetsbaarheden aan te pakken, omvat vulnerability management een breder scala aan activiteiten, waaronder configuratiebeheer, beveiligingsbewustzijnstraining en penetratietesten, om de algehele beveiliging te verbeteren en risico's te beheren.

 \autocite{Danby2023}

\section{Trends en ontwikkelingen in SAP ERP en patchmanagement}
Recente ontwikkelingen binnen SAP ERP hebben geleid tot een aantal significante trends op het gebied van patchmanagement.

 Een van de meest opvallende trends is de verschuiving naar meer geautomatiseerde en geïntegreerde patchmanagementprocessen.

 Dit houdt in dat patches steeds vaker automatisch worden gedetecteerd, geëvalueerd en toegepast, waardoor het patchen van systemen efficiënter en minder arbeidsintensief wordt.

 
Een tweede belangrijke trend is de nadruk op proactieve beveiligingsmaatregelen.

 In plaats van te wachten tot zich kwetsbaarheden voordoen, streven organisaties ernaar om potentiële beveiligingsrisico's voor te zijn door regelmatig patches toe te passen en beveiligingsupdates te implementeren.

 Deze verschuiving naar proactieve beveiliging wordt verder aangewakkerd door de groeiende dreiging van cyberaanvallen en de noodzaak om de beveiliging van bedrijfskritieke systemen te waarborgen, vooral nu veel werknemers op afstand werken.

 
Nog een belangrijke trend is dat ontwikkelingen binnen SAP ERP hebben geleid tot een groeiende afhankelijkheid van cloudservices voor patchmanagement.

 Steeds meer organisaties verplaatsen hun ERP-systemen naar de cloud om te profiteren van schaalbaarheid, flexibiliteit en lagere operationele kosten.

 Deze verschuiving naar de cloud heeft ook invloed op patchmanagement, aangezien organisaties nu op zoek zijn naar cloudgebaseerde oplossingen voor het beheren en implementeren van patches over hun gedistribueerde netwerken en apparaten.

 \autocite{Kannan2023}
\section{Uitdagingen bij patching}
Organisaties worden geconfronteerd met verschillende uitdagingen bij het beheren van patches.

 Een van de meest voorkomende problemen is de traagheid waarmee updates worden uitgevoerd.

 Dit kan te wijten zijn aan verschillende factoren, waaronder het gebrek aan middelen of prioriteitstoewijzing binnen de IT-afdeling.

 Als gevolg hiervan blijven systemen mogelijk kwetsbaar voor bekende beveiligingsrisico's, wat de algehele veiligheid van het bedrijf in gevaar kan brengen.

 \autocite{AppMaster2023}
Ook werken veel organisaties met meerdere systemen en toepassingen.

 Dit betekent dat IT-professionals verschillende besturingssystemen moeten kunnen patchen, zoals Linux, MacOS en Windows, evenals verschillende applicaties van derden.

 Het patchen van al deze systemen kan dus een hele klus zijn.


Het patchen van systemen wordt vaak als tijdrovend en complex ervaren door IT- en beveiligingsprofessionals.

 Volgens \textcite{ivanti2021} vindt maar liefst 71\% van hen patchen te complex en tijdrovend.

 Het handmatig patchen van elk apparaat op een netwerk is niet alleen frustrerend en langzaam, maar ook buitengewoon inefficiënt voor grotere organisaties.

 Dit leidt tot grote kosten voor bedrijven die veel moeten patchen, waardoor sommigen beroep doen op het Aziatische landen waar de kostprijs per werkuur veel lager ligt.

 
Aangezien ERP-systemen vaak complexe architecturen hebben en geïntegreerd zijn met een breed scala aan andere systemen en applicaties, kunnen patches soms onverwachte problemen veroorzaken.

 Dit kan leiden tot verstoring van bedrijfsactiviteiten en downtime, wat op zijn beurt kan leiden tot financiële verliezen en reputatieschade voor het bedrijf.


	
\section{De keuze tussen handmatig en automatisch patchen}
Binnen organisaties die gebruikmaken van SAP ERP-systemen, staan IT-teams vaak voor de uitdaging om te beslissen tussen handmatig patchen en automatisch patchen.

 Elk van deze benaderingen heeft zijn eigen voor- en nadelen, en het is cruciaal voor organisaties om de juiste keuze te maken op basis van hun specifieke behoeften en omgeving.


Handmatig patchen biedt IT-teams een hoog niveau van controle en flexibiliteit geboden bij het beheren van patches binnen SAP ERP.

 Door middel van deze aanpak kan elk patch, klant of server-combinatie afzonderlijk worden behandeld en kan alleen geïnstalleerd worden wat nodig is, wanneer dat nodig is, waardoor het risico op onbedoelde gevolgen wordt verminderd.

 Bovendien kan door IT-teams handmatige patches aangepast worden aan de unieke behoeften van hun organisatie, waardoor compatibiliteitsproblemen vermeden kunnen worden en de algehele stabiliteit van het ERP-systeem gewaarborgd kan worden.

 \autocite{Hooper2018}
Het nadeel van handmatig patchen is dat het tijdsintensief is en vereist aanzienlijke menselijke middelen.

 IT-teams moeten voortdurend patchbulletins volgen, patches testen en implementeren, wat kan leiden tot vertragingen bij het toepassen van cruciale updates.

 Bovendien vergroot de menselijke factor het risico op fouten, waardoor organisaties kwetsbaar kunnen zijn voor beveiligingsrisico's als gevolg van gemiste patches of onjuiste implementaties.


Aan de andere kant biedt automatisch patchen een hoog niveau van efficiëntie en consistentie bij het beheren van patches binnen SAP ERP.

 Met automatische patchingtools kunnen patches automatisch worden gedetecteerd, gedownload en toegepast volgens een vooraf ingesteld schema, waardoor de kans op gemiste patches wordt geminimaliseerd en de algehele beveiliging van het systeem wordt versterkt.

 Bovendien kunnen automatische patchingtools zorgen voor een snelle respons op nieuwe beveiligingsdreigingen, wat de kwetsbaarheid van het systeem vermindert.


Echter, automatisch patchen vereist een betrouwbare en goed geconfigureerde tool om effectief te werken.

 Organisaties moeten ervoor zorgen dat ze de juiste patchingtools hebben geïmplementeerd en dat deze zijn afgestemd op hun specifieke behoeften en omgeving.

 Daarnaast bestaat het risico dat automatische patchingtools incompatibele patches implementeren, wat kan leiden tot systeemstoringen of conflicten met andere softwarecomponenten.



Volgens \textcite{Tozzi2017} is het beste om een evenwichtige update-strategie te hanteren.

 Deze strategie moet gebruikmaken van automatische updates voor zover ze nuttig zijn, maar automatische updates vermijden in gevallen waarin ze te veel risico met zich meebrengen of onvoldoende zijn.

 Update-strategieën moeten natuurlijk worden afgestemd op uw specifieke behoeften.

 Over het algemeen zal een goed ontworpen update-strategie zich richten op automatische updates voor kritieke beveiligingslekken, updates voor het besturingssysteem zelf, en updates voor systemen die gemakkelijk kunnen worden teruggerold.

 Aan de andere kant worden updates voor firmware, randapparatuur, netwerkschakelaars en andere soorten software die niet betrouwbaar kunnen worden geautomatiseerd, samen met niet-kritieke updates en updates voor systemen die zeer beschikbaar moeten zijn, het beste gereserveerd voor handmatige patching.

 Deze benadering kan u helpen om het juiste evenwicht te vinden tussen up-to-date en veilig zijn aan de ene kant, en uw software stabiel houden aan de andere kant.



\section{Invloed van menselijke, technologische en organisatorische factoren op patchmanagement}
Patchmanagement binnen SAP ERP-systemen wordt sterk beïnvloed door een combinatie van menselijke, technologische en organisatorische factoren.

 Een diepgaande analyse van deze elementen is essentieel om een effectieve patchmanagementstrategie te ontwikkelen en te implementeren.


Menselijke factoren spelen een cruciale rol in het patchmanagementproces.

 Het bewustzijn van medewerkers over de noodzaak van patching en het belang van cybersecurity is van vitaal belang.

 Zonder voldoende bewustwording kunnen patches over het hoofd worden gezien of niet tijdig worden toegepast, waardoor de veiligheid van het ERP-systeem in gevaar komt.

 Daarnaast is training van medewerkers essentieel om ervoor te zorgen dat ze de juiste procedures volgen bij het beoordelen, testen en implementeren van patches.

 Een goed opgeleid team kan de efficiëntie van het patchmanagementproces aanzienlijk verbeteren en de kans op menselijke fouten minimaliseren.



Technologische factoren omvatten de beschikbaarheid en kwaliteit van patchingtools en -technologieën binnen SAP ERP.

 Geavanceerde patchingtools kunnen het patchmanagementproces automatiseren en vereenvoudigen, waardoor de werklast voor IT-teams wordt verminderd en de kans op fouten wordt verkleind.

 Het gebruik van geautomatiseerde tools kan ook zorgen voor een snellere detectie en reactie op beveiligingsdreigingen, wat cruciaal is voor het beschermen van het ERP-systeem tegen potentiële aanvallen.



Organisatorische factoren, zoals de samenwerking tussen verschillende afdelingen en de implementatie van duidelijke patchmanagementprocedures, zijn ook van groot belang.

 Een goed gestructureerd patchmanagementbeleid, ondersteund door heldere communicatiekanalen en duidelijke verantwoordelijkheden, kan de efficiëntie van het patchmanagementproces aanzienlijk verbeteren.

 Door nauw samen te werken met verschillende belanghebbenden, waaronder IT-teams, beveiligingsexperts en operationele afdelingen, kunnen organisaties ervoor zorgen dat patching prioriteit krijgt en effectief wordt uitgevoerd.



In het specifieke geval van SAP ERP is het belangrijk om te benadrukken dat patchmanagement niet alleen een technische kwestie is, maar ook een strategisch aspect van IT-beveiliging.

 Het succesvol beheren van patches vereist daarom een holistische benadering waarbij rekening wordt gehouden met zowel menselijke, technologische als organisatorische factoren.

 Door bewustwording te vergroten, medewerkers te trainen, geavanceerde tools te gebruiken en effectieve samenwerking te bevorderen, kunnen organisaties hun SAP ERP-systemen adequaat beschermen tegen potentiële bedreigingen en de operationele continuïteit waarborgen.

 \autocite{Graffeo2018}

\section{Strategieën voor minimalisering van de impact op de continuïteit:}
Effectieve patching vereist zorgvuldige planning om de impact te minimaliseren.

 Het is raadzaam patchactiviteiten te plannen tijdens niet-kritieke operationele periodes.

 Daarnaast kan een gefaseerde implementatie, beginnend met minder kritieke systemen, de operationele continuïteitswaarborgen.

 Het implementeren van fallbackstrategieën is essentieel voor een snelle en efficiënte herstelprocedure in geval van onvoorziene problemen.

 Patching kan falen door verschillende redenen, variërend van incompatibele hardware tot conflicten met andere patches, of een patch die goed installeert maar iets anders kapotmaakt \autocite{Shein2022}.


Het implementeren van patches binnen SAP ERP-systemen kan een aanzienlijke invloed hebben op de operationele continuïteit van een organisatie.

 Om deze impact te minimaliseren, zijn er verschillende strategieën die organisaties kunnen toepassen tijdens het patchmanagementproces.


Ten eerste is zorgvuldige planning essentieel.

 Het is belangrijk om patchactiviteiten te plannen tijdens niet-kritieke operationele periodes, zoals buiten de reguliere kantooruren of tijdens perioden van verminderde bedrijfsactiviteit.

 Door patches op strategische momenten toe te passen, kan de verstoring van lopende processen tot een minimum worden beperkt.


Daarnaast is een gefaseerde implementatie van patches een effectieve strategie om de impact op de continuïteit te minimaliseren.

 In plaats van alle systemen tegelijkertijd te patchen, kunnen organisaties ervoor kiezen om patches geleidelijk aan te brengen, te beginnen met minder kritieke systemen.

 Op deze manier kunnen eventuele problemen of compatibiliteitsproblemen worden geïdentificeerd en opgelost voordat de patches op cruciale systemen worden toegepast.


Het opzetten van fallbackstrategieën is ook essentieel voor een succesvol patchmanagementproces.

 Ondanks zorgvuldige planning en gefaseerde implementatie kunnen er onvoorziene problemen optreden tijdens het patchen, zoals systeemstoringen of conflicten tussen patches.

 Organisatieskstrategieën te definiëren en vooraf te plannen, kunnen organisaties snel en efficiënt reageren op dergelijke situaties en de operationele continuïteit waarborgen.

 Dit kan bijvoorbeeld het creëren van back-ups van cruciale systemen omvatten, zodat ze snel kunnen worden hersteld in geval van problemen tijdens het patchen.



In het kort benadrukken zorgvuldige planning, gefaseerde implementatie en fallbackstrategieën het belang van een proactieve en strategische benadering van patchmanagement.

 Door deze strategieën toe te passen, kunnen organisaties de impact van patchactiviteiten op de operationele continuïteit minimaliseren en tegelijkertijd de veiligheid en stabiliteit van hun SAP ERP-systemen waarborgen.



\section{Communicatiestrategieën voor belanghebbenden}
Heldere en tijdige communicatie met belanghebbenden over geplande patchactiviteiten is van vitaal belang.

 Het is essentieel om begrijpelijke informatie te verstrekken over de redenen achter het patchen, mogelijke impact en benodigde voorzorgsmaatregelen \autocite{Toren2019}.

 
Daarnaast zijn het opzetten van communicatiekanalen voor feedback en het bieden van ondersteuning aan gebruikers eveneens cruciaal voor een succesvolle implementatie.



\section{Patchmanagement in cloudgebaseerde ERP-systemen}
Volgens \textcite{Forbes2021} kan de cloud een meer gestroomlijnde benadering bieden, terwijl tegelijkertijd geprofiteerd wordt van beveiligings best practices en updates van cloud- en softwareleveranciers.

 Volgens \textcite{Cox2020} bieden cloudgebaseerde ERP-systemen verschillende voordelen  ten opzichte van traditionele on-premises oplossingen.

 Ze kunnen gemakkelijk meeschalen met de groeiende behoeften van een organisatie, vereenvoudigen het implementatieproces van beveiligingspatches en bieden realtime zichtbaarheid in de patchstatus van alle apparaten, ongeacht hun locatie.

 De initiële kosten zijn lager bij de cloudinfrastructuur, en deze kosten zijn ook meer voorspelbaar, terwijl bij een on-premise ERP-systeem de initiële kosten veel hoger kunnen zijn, maar de totale kosten na verloop van tijd lager kunnen uitvallen, afhankelijk van de vereisten.

 Een ander voordeel van de cloudprovider is dat de verantwoordelijkheid voor data beveiliging nu ook bij hen ligt.


De interesse in ERP overstijgt verticale industrieën, met fabrikanten, dienstverleners, non-profitorganisaties en overheidsentiteiten die allemaal behoefte hebben aan de mogelijkheden ervan om efficiënt en effectief te kunnen functioneren.

 In de afgelopen jaren zijn velen overgestapt van decennia-oude on-premises ERP-systemen naar nieuwe cloudgebaseerde ERPs.

 Anderen die nooit ERP hadden, gaan rechtstreeks naar de cloudoptie, vooral SaaS ERP.


We zien ook dat de markt dit reflecteert en dat nu minder dan een derde van de bedrijfsapplicaties naar verwachting in 2022 nog gehost zal worden op traditionele servers, terwijl de rest zal vertrouwen op cloud computing-oplossingen.

 Bedrijven over de hele wereld vertrouwen nu steeds meer op cloud computing om hun bedrijf te runnen.

 Het groeit zo snel in populariteit dat minder dan een derde van de bedrijfsapplicaties nog verwacht wordt te worden gehost op traditionele servers tegen 2022, en de rest zal vertrouwen op cloud computing-oplossingen \autocite{Pimentel2017}.


Patchmanagement in cloudgebaseerde ERP-systemen is dus een steeds belangrijker wordend aspect van IT-beheer voor moderne organisaties.

 Met de opkomst van cloud computing hebben veel bedrijven ervoor gekozen om hun ERP-systemen naar de cloud te verplaatsen om te profiteren van de schaalbaarheid, flexibiliteit en kosteneffectiviteit die de cloud biedt.

 Echter, het beheren van patches in een cloudomgeving brengt unieke uitdagingen met zich mee.


Om succesvol patchmanagement in cloudgebaseerde ERP-systemen te implementeren, moeten organisaties investeren in geautomatiseerde patchingtools, beleidsregels en procedures voor patchmanagement, en continue monitoring en rapportage van patch statussen.

 Daarnaast is samenwerking en communicatie tussen IT-teams, cloudproviders en andere belanghebbenden van vitaal belang om ervoor te zorgen dat patches tijdig worden geïmplementeerd zonder de operationele continuïteit te verstoren.

 Met de juiste aanpak kunnen organisaties de beveiliging en betrouwbaarheid van hun cloudgebaseerde ERP-systemen waarborgen in een snel evoluerend technologisch landschap.



\section{De toekomst van patchmanagement}
De toekomst van AI in patchmanagement ziet er veelbelovend uit, vooral gezien de groeiende uitdagingen waarmee organisaties te maken hebben bij het beheren van patches.

 Het aantal fixes dat wekelijks wordt uitgegeven, neemt toe, waardoor het moeilijk wordt om te bepalen welke patches moeten worden toegepast, hoe snel en waar.

 Dit brengt een grote werklast met zich mee voor patchmanagementprogramma's, die exponentieel toeneemt naarmate organisaties het aantal endpoints en de variëteit aan softwarebronnen uitbreiden.


Om deze uitdagingen aan te pakken, wenden steeds meer bedrijven zich tot AI en machine learning.

 Deze technologieën kunnen worden gebruikt om patches te detecteren, prioriteren en snel toe te passen wanneer dat nodig is.

 Dit resulteert in een efficiëntieverhoging die de algehele beveiliging kan verbeteren door kwetsbaarheden eerder te detecteren en de kans te verkleinen dat ze worden misbruikt in aanvallen in het echte leven.


AI-algoritmen begrijpen de complexe relaties tussen concurrerende variabelen en kunnen een patch schema aanbevelen dat is afgestemd op de specifieke behoeften van een organisatie.

 Bovendien kunnen AI-tools helpen bij het verminderen van compatibiliteitsrisico's door intelligente implementatietests uit te voeren en de belasting van IT-resources te verminderen.

 Met behulp van AI kunnen bedrijven ook endpoint en gebruikersprofielen evalueren, zodat alleen relevante patches op het juiste moment worden toegepast, met minimale impact op gebruikers en bedrijfsactiviteiten.


Hoewel de toepassing van AI in patchmanagement veelbelovend is, zijn er ook uitdagingen en risico's.

 Het gebruik van AI in patchmanagement staat nog in de kinderschoenen, en er is een steile leercurve voor organisaties om deze nieuwe technologie effectief te implementeren.

 Er zijn ook ethische overwegingen bij het gebruik van AI voor autonome beslissingen over patchprioritering en implementatie.

 Bovendien is AI niet perfect en zijn de voorspellingen niet altijd nauwkeurig, wat kan leiden tot fouten bij het beoordelen van de impact van patches.

 \autocite{OFlaherty2023}

\section{Patchmanagement best practices}
Om patchmanagement efficiënt te implementeren binnen uw onderneming, zijn er enkele cruciale stappen die moeten worden gevolgd.

 Allereerst is het van essentieel belang om een gecentraliseerd overzicht te hebben van alle systemen in het netwerk, waardoor prioriteiten kunnen worden gesteld voor ontbrekende patches op basis van hun ernst.

 Vervolgens moeten patchdeployments worden gepland zonder de productiviteit van werknemers te beïnvloeden, waarbij gebruik wordt gemaakt van tools om automatisch patches te implementeren op basis van de beschikbaarheid van gebruikers en de uptime van systemen.

 Daarnaast moeten strategieën worden ontwikkeld om alle systemen binnen het netwerk te patchen, inclusief LAN, externe kantoren en thuiswerkende werknemers.

 Het testen van patches voor implementatie en het hebben van de mogelijkheid om patches terug te draaien in geval van problemen zijn ook essentiële stappen voor een efficiënt patchmanagementproces.

 Tot slot is het van belang om een centrale patchmanagementoplossing te kiezen, patches voor implementatie te prioriteren, patching te automatiseren en het volgen en genereren van rapporten over de patch-deploying in het netwerk te waarborgen.

 \autocite{ManageEngine2024}