\chapter{\IfLanguageName{dutch}{Stand van zaken}{State of the art}}
\label{ch:stand-van-zaken}

% Tip: Begin elk hoofdstuk met een inleidende paragraaf die beschrijft hoe
% dit hoofdstuk past binnen het geheel van de bachelorproef.
% Geef in het bijzonder aan wat de link is met het vorige en volgende hoofdstuk.

In dit hoofdstuk wordt de stand van zaken met betrekking tot ERP-systemen en patchmanagement behandeld. De focus ligt op het belang van patchmanagement binnen ERP-systemen, met name SAP ERP, en de uitdagingen en strategieën die gepaard gaan met het effectief beheren van patches in deze omgeving.

\section{Inleiding tot ERP-systemen en patchmanagement}
Enterprise Resource Planning (ERP) systemen vormen de ruggengraat van moderne bedrijfsvoering. Deze geïntegreerde softwaretoepassingen spelen een essentiële rol bij het stroomlijnen en automatiseren van verschillende bedrijfsprocessen, 

variërend van financiën en voorraadbeheer tot human resources en klantrelatiebeheer.

Door het centraliseren van gegevens en processen helpen ERP-systemen organisaties om efficiënter te werken, betere beslissingen te nemen en hun concurrentiepositie te versterken \autocite{StatistiekVlaanderen2022}.

Met de toenemende afhankelijkheid van ERP-systemen komt ook de groeiende dreiging van cyberaanvallen. Als centrale hubs voor gevoelige bedrijfsgegevens, worden ERP-systemen vaak het doelwit van kwaadwillende actoren die uit zijn op gegevensdiefstal, 

verstoring van bedrijfsprocessen of financieel gewin. In deze context wordt het belang van patchmanagement binnen ERP-systemen steeds groter, vooral om de veiligheid en stabiliteit van de systemen te waarborgen \autocite{Pearson2024}.

Het implementeren van een ERP-systeem is een cruciale stap voor organisaties van een bepaalde omvang, meestal ergens rond een paar honderd werknemers. Zelfs sommige kleinere organisaties met complexe activiteiten zullen de noodzaak van ERP vinden.

De vraag naar ERP heeft een aanzienlijke markt gecreëerd. Volgens \textcite{Madh2024} was de wereldwijde ERP-softwaremarkt in 2023 \$49,80 miljard en wordt verwacht dat deze tegen 2030 zal stijgen naar \$140,14 miljard. Drie grote namen - 

Microsoft, Oracle en SAP - domineren de markt, maar verschillende kleinere spelers bieden ook ERP-producten aan die op vele manieren concurrerend zijn met die van de marktleiders \autocite{Pratt2023}.

Patchmanagement verwijst naar het proces van het identificeren, beoordelen, testen en implementeren van softwarepatches om bekende kwetsbaarheden in systemen te verhelpen. Binnen het domein van ERP-systemen, met name binnen SAP ERP,

is patchmanagement van cruciaal belang om de veiligheid en stabiliteit van de systemen te waarborgen. Door regelmatig patches toe te passen, kunnen organisaties potentiële beveiligingsrisico's minimaliseren en zichzelf beschermen tegen externe bedreigingen \autocite{Buenning2024}.

\section{Soorten patches}
Beveiligingspatches, die het aanpakken van nieuw ontdekte beveiligingskwetsbaarheden in het systeem inhouden, zijn een cruciaal aspect van het handhaven van de integriteit en beveiliging van elke softwareomgeving. Naast beveiligingspatches spelen

bugfixes en prestatiepatches een essentiële rol bij het zorgen voor de soepele werking en het verbeteren van de algehele prestaties van softwaresystemen. Deze verschillende soorten patches dragen gezamenlijk bij aan de stabiliteit, 

beveiliging en efficiëntie van softwaresystemen \autocite{Wrobel2023}.

\section{Patchmanagement vs Vulnerability management}
Beveiligingspatches, die het aanpakken van nieuw ontdekte beveiligingskwetsbaarheden in het systeem inhouden, zijn een cruciaal aspect van het handhaven van de integriteit en beveiliging van elke softwareomgeving.

Naast beveiligingspatches spelen bugfixes een essentiële rol bij het zorgen voor de soepele werking van softwaresystemen door problemen of fouten aan te pakken die zich kunnen voordoen in de functionaliteit van het systeem.

Daarnaast zijn prestatiepatches essentieel voor het verbeteren van de algehele prestaties van het systeem.

Deze patches zijn bedoeld om de resource-eisen te verminderen, de snelheid van toepassingen te verbeteren en nieuwe functionaliteiten te introduceren om de gebruikerservaring en efficiëntie te optimaliseren.

Deze verschillende soorten patches dragen gezamenlijk bij aan de stabiliteit, beveiliging en efficiëntie van softwaresystemen. Het is van cruciaal belang voor organisaties om regelmatig deze patches te beoordelen en te implementeren om risico's te 

beperken en de soepele werking van hun systemen te waarborgen.

Deze concepten worden weerspiegeld in het onderzoek van \textcite{Wrobel2023}, waarin het belang van patchmanagementstrategieën wordt benadrukt bij het aanpakken van beveiligingskwetsbaarheden,

bugs en prestatieproblemen binnen softwareomgevingen.

\section{Trends en ontwikkelingen in SAP ERP en patchmanagement}
Recente ontwikkelingen binnen SAP ERP hebben geleid tot een aantal significante trends op het gebied van patchmanagement.

Een van de meest opvallende trends is de verschuiving naar meer geautomatiseerde en geïntegreerde patchmanagementprocessen.

Dit houdt in dat patches steeds vaker automatisch worden gedetecteerd, geëvalueerd en toegepast, waardoor het patchen van systemen efficiënter en minder arbeidsintensief wordt.

Een tweede belangrijke trend is de nadruk op proactieve beveiligingsmaatregelen.

In plaats van te wachten tot zich kwetsbaarheden voordoen, streven organisaties ernaar om potentiële beveiligingsrisico's voor te zijn door regelmatig patches toe te passen en beveiligingsupdates te implementeren.

Nog een belangrijke trend is dat ontwikkelingen binnen SAP ERP hebben geleid tot een groeiende afhankelijkheid van cloudservices voor patchmanagement.

Steeds meer organisaties verplaatsen hun ERP-systemen naar de cloud om te profiteren van schaalbaarheid, flexibiliteit en lagere operationele kosten.

Deze verschuiving naar de cloud heeft ook invloed op patchmanagement, aangezien organisaties nu op zoek zijn naar cloudgebaseerde oplossingen voor het beheren en implementeren van patches over hun gedistribueerde netwerken en apparaten.

\section{Uitdagingen bij patching}
Organisaties worden geconfronteerd met verschillende uitdagingen bij het beheren van patches.

Een van de meest voorkomende problemen is de traagheid waarmee updates worden uitgevoerd.

Dit kan te wijten zijn aan verschillende factoren, waaronder het gebrek aan middelen of prioriteitstoewijzing binnen de IT-afdeling.

Als gevolg hiervan blijven systemen mogelijk kwetsbaar voor bekende beveiligingsrisico's, wat de algehele veiligheid van het bedrijf in gevaar kan brengen.

Ook werken veel organisaties met meerdere systemen en toepassingen.

Dit betekent dat IT-professionals verschillende besturingssystemen moeten kunnen patchen, zoals Linux, MacOS en Windows, evenals verschillende applicaties van derden.

Het patchen van al deze systemen kan dus een hele klus zijn.

Het patchen van systemen wordt vaak als tijdrovend en complex ervaren door IT- en beveiligingsprofessionals.

Volgens \textcite{ivanti2021} vindt maar liefst 71\% van hen patchen te complex en tijdrovend.

Het handmatig patchen van elk apparaat op een netwerk is niet alleen frustrerend en langzaam, maar ook buitengewoon inefficiënt voor grotere organisaties.

Dit leidt tot grote kosten voor bedrijven die veel moeten patchen, waardoor sommigen beroep doen op landen waar de kostprijs per werkuur veel lager ligt.

Aangezien ERP-systemen vaak complexe architecturen hebben en geïntegreerd zijn met een breed scala aan andere systemen en applicaties, kunnen patches soms onverwachte problemen veroorzaken.

Dit kan leiden tot verstoring van bedrijfsactiviteiten en downtime, wat op zijn beurt kan leiden tot financiële verliezen en reputatieschade voor het bedrijf.


\section{De keuze tussen handmatig en automatisch patchen}

Binnen organisaties die gebruikmaken van SAP ERP-systemen, staan IT-teams vaak voor de uitdaging om te beslissen tussen handmatig patchen en automatisch patchen.

Elk van deze benaderingen heeft zijn eigen voor- en nadelen, en het is cruciaal voor organisaties om de juiste keuze te maken op basis van hun specifieke behoeften en omgeving.

Handmatig patchen biedt IT-teams een hoog niveau van controle en flexibiliteit bij het beheren van patches binnen SAP ERP. Door middel van deze aanpak kan elk patch,

klant of server-combinatie afzonderlijk worden behandeld en kan alleen geïnstalleerd worden wat nodig is, wanneer dat nodig is, waardoor het risico op onbedoelde gevolgen wordt verminderd.

Bovendien kunnen handmatige patches worden aangepast aan de unieke behoeften van een organisatie, waardoor compatibiliteitsproblemen kunnen worden vermeden en de algehele 

stabiliteit van het ERP-systeem kan worden gewaarborgd \autocite{Hooper2018}. 

Het nadeel van handmatig patchen is dat het tijdsintensief is en aanzienlijke menselijke middelen vereist. IT-teams moeten voortdurend patchbulletins volgen, 

patches testen en implementeren, wat kan leiden tot vertragingen bij het toepassen van cruciale updates.

Bovendien vergroot de menselijke factor het risico op fouten, waardoor organisaties kwetsbaar kunnen zijn voor beveiligingsrisico's als gevolg van gemiste patches of onjuiste implementaties.

Aan de andere kant biedt automatisch patchen een hoog niveau van efficiëntie en consistentie bij het beheren van patches binnen SAP ERP. 

Met automatische patchingtools kunnen patches automatisch worden gedetecteerd, gedownload en toegepast volgens een vooraf ingesteld schema, 

waardoor de kans op gemiste patches wordt geminimaliseerd en de algehele beveiliging van het systeem wordt versterkt. 

Bovendien kunnen automatische patchingtools zorgen voor een snelle respons op nieuwe beveiligingsdreigingen, wat de kwetsbaarheid van het systeem vermindert. 

Echter, automatisch patchen vereist een betrouwbare en goed geconfigureerde tool om effectief te werken. Organisaties moeten ervoor zorgen dat ze de juiste patchingtools

hebben geïmplementeerd en dat deze zijn afgestemd op hun specifieke behoeften en omgeving. Daarnaast bestaat het risico dat automatische patchingtools incompatibele patches implementeren,

wat kan leiden tot systeemstoringen of conflicten met andere softwarecomponenten.

Volgens \textcite{Tozzi2017} is het het beste om een evenwichtige updatestrategie te hanteren. Deze strategie moet gebruikmaken van automatische updates voor zover ze nuttig zijn,

maar automatische updates vermijden in gevallen waarin ze te veel risico met zich meebrengen of onvoldoende zijn. Updatestrategieën moeten natuurlijk worden afgestemd op uw specifieke behoeften. 

Over het algemeen zal een goed ontworpen updatestrategie zich richten op automatische updates voor kritieke beveiligingslekken, updates voor het besturingssysteem zelf,

en updates voor systemen die gemakkelijk kunnen worden teruggerold. Aan de andere kant worden updates voor firmware, randapparatuur,

netwerkschakelaars en andere soorten software die niet betrouwbaar kunnen worden geautomatiseerd, samen met niet-kritieke updates en updates voor systemen die zeer beschikbaar moeten zijn,

het beste gereserveerd voor handmatige patching. Deze benadering kan u helpen om het juiste evenwicht te vinden tussen up-to-date en veilig zijn aan de ene kant, en uw software stabiel houden aan de andere kant.

