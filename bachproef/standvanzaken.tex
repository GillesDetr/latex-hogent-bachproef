\chapter{\IfLanguageName{dutch}{Stand van zaken}{State of the art}}
\label{ch:stand-van-zaken}

% Tip: Begin elk hoofdstuk met een inleidende paragraaf die beschrijft hoe
% dit hoofdstuk past binnen het geheel van de bachelorproef.
% Geef in het bijzonder aan wat de link is met het vorige en volgende hoofdstuk.

In dit hoofdstuk wordt de stand van zaken met betrekking tot ERP-systemen en patchmanagement behandeld. De focus ligt op het belang van patchmanagement binnen ERP-systemen, met name SAP ERP, en de uitdagingen en strategieën die gepaard gaan met het effectief beheren van patches in deze omgeving.

\section{Inleiding tot ERP-systemen en patchmanagement}
Enterprise Resource Planning (ERP) systemen vormen de ruggengraat van moderne bedrijfsvoering. Deze geïntegreerde softwaretoepassingen spelen een essentiële rol bij het stroomlijnen en automatiseren van verschillende bedrijfsprocessen, variërend van financiën en voorraadbeheer tot human resources en klantrelatiebeheer. Door het centraliseren van gegevens en processen helpen ERP-systemen organisaties om efficiënter te werken, betere beslissingen te nemen en hun concurrentiepositie te versterken \autocite{StatistiekVlaanderen2022}.

Met de toenemende afhankelijkheid van ERP-systemen komt ook de groeiende dreiging van cyberaanvallen. Als centrale hubs voor gevoelige bedrijfsgegevens, worden ERP-systemen vaak het doelwit van kwaadwillende actoren die uit zijn op gegevensdiefstal, verstoring van bedrijfsprocessen of financieel gewin. In deze context wordt het belang van patchmanagement binnen ERP-systemen steeds groter, vooral om de veiligheid en stabiliteit van de systemen te waarborgen \autocite{Pearson2024}.

Het implementeren van een ERP-systeem is een cruciale stap voor organisaties van een bepaalde omvang, meestal ergens rond een paar honderd werknemers. Zelfs sommige kleinere organisaties met complexe activiteiten zullen de noodzaak van ERP vinden.

De vraag naar ERP heeft een aanzienlijke markt gecreëerd. Volgens \textcite{Madh2024} was de wereldwijde ERP-softwaremarkt in 2023 \$49,80 miljard en wordt verwacht dat deze tegen 2030 zal stijgen naar \$140,14 miljard. Drie grote namen - Microsoft, Oracle en SAP - domineren de markt, maar verschillende kleinere spelers bieden ook ERP-producten aan die op vele manieren concurrerend zijn met die van de marktleiders \autocite{Pratt2023}.

Patchmanagement verwijst naar het proces van het identificeren, beoordelen, testen en implementeren van softwarepatches om bekende kwetsbaarheden in systemen te verhelpen. Binnen het domein van ERP-systemen, met name binnen SAP ERP, is patchmanagement van cruciaal belang om de veiligheid en stabiliteit van de systemen te waarborgen. Door regelmatig patches toe te passen, kunnen organisaties potentiële beveiligingsrisico's minimaliseren en zichzelf beschermen tegen externe bedreigingen \autocite{Buenning2024}.

\section{Soorten patches}
Beveiligingspatches, die het aanpakken van nieuw ontdekte beveiligingskwetsbaarheden in het systeem inhouden, zijn een cruciaal aspect van het handhaven van de integriteit en beveiliging van elke softwareomgeving. Naast beveiligingspatches spelen bugfixes en prestatiepatches een essentiële rol bij het zorgen voor de soepele werking en het verbeteren van de algehele prestaties van softwaresystemen. Deze verschillende soorten patches dragen gezamenlijk bij aan de stabiliteit, beveiliging en efficiëntie van softwaresystemen \autocite{Wrobel2023}.

\section{Patchmanagement vs Vulnerability management}
Patchmanagement richt zich op het identificeren, implementeren en beheren van software-updates (patches) om bekende beveiligingslekken in systemen en applicaties
