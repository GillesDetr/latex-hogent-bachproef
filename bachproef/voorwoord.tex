%%=============================================================================
%% Voorwoord
%%=============================================================================

\chapter*{\IfLanguageName{dutch}{Woord vooraf}{Preface}}%
\label{ch:voorwoord}

%% TODO:
%% Het voorwoord is het enige deel van de bachelorproef waar je vanuit je
%% eigen standpunt (``ik-vorm'') mag schrijven. Je kan hier bv. motiveren
%% waarom jij het onderwerp wil bespreken.
%% Vergeet ook niet te bedanken wie je geholpen/gesteund/... heeft


Als student Toegepaste Informatica aan Hogeschool Gent, wil ik onder andere steeds opzoek gaan naar nieuwe manieren om processen te optimaliseren en te automatiseren. Dit is dan ook aansluitend bij mijn onderzoek. Het begrijpen van hoe ERP-systemen functioneren en hoe ze worden onderhouden, is ook van belang bij het bevorderen van mijn professionele ontwikkeling.

Omdat ERP-systemen een belangrijk onderdeel zijn van de moderne bedrijfsvoering, sprak het onderwerp “patchmanagement optimalisatie binnen ERP-bedrijven” me dan ook enorm aan om mijn bachelorproef hierover te maken.

Door de opkomst van de groeiende complexiteit van IT-infrastructuren, is het essentieel dat we alle processen kunnen begrijpen en dan ook te kunnen optimaliseren.

Ik wil graag mijn oprechte dank betuigen aan de mensen die me ondersteund hebben tijdens dit proces. Zo wil ik onder andere Kasper Heyndrickx, mijn co-promotor, bedanken voor de waardevolle feedback en begeleiding tijdens het volledige proces van deze bachelorproef.

Ook wil ik Miguel De Munck en Olivier Troch bedanken voor hun ondersteuning en inzet bij het uitvoeren van de case study. Hun expertises en adviezen hebben waardevolle bijdragen geleverd in dit onderzoek.

%% Ik wil graag mijn oprechte dank betuigen aan mijn promotor, Mark Asselberg, voor zijn waardevolle begeleiding en ondersteuning tijdens het schrijven van deze bachelorproef. Zijn inzichten en feedback hebben mij geholpen om mijn onderzoek naar een hoger niveau te tillen en een dieper inzicht te krijgen in het onderwerp.