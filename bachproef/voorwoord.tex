%%=============================================================================
%% Voorwoord
%%=============================================================================

\chapter*{\IfLanguageName{dutch}{Woord vooraf}{Preface}}%
\label{ch:voorwoord}

%% TODO:
%% Het voorwoord is het enige deel van de bachelorproef waar je vanuit je
%% eigen standpunt (``ik-vorm'') mag schrijven. Je kan hier bv. motiveren
%% waarom jij het onderwerp wil bespreken.
%% Vergeet ook niet te bedanken wie je geholpen/gesteund/... heeft


Ik heb dit onderwerp gekozen omdat ik ERP-systemen zie als een integraal onderdeel van moderne bedrijfsvoering. In een tijd waarin digitalisering en efficiëntie centraal staan, spelen ERP-systemen een cruciale rol bij het stroomlijnen van bedrijfsprocessen.

Als student in Toegepaste Informatica, ben ik gefascineerd door de technologische ontwikkelingen die de bedrijfswereld transformeren. Het begrijpen van hoe ERP-systemen functioneren en hoe ze worden beheerd en onderhouden, is van groot belang voor mijn professionele ontwikkeling.

Daarnaast ben ik geïnteresseerd in de uitdagingen en mogelijkheden die gepaard gaan met het patchmanagement van cloudgebaseerde ERP-systemen. Met de opkomst van de cloud en de groeiende complexiteit van IT-infrastructuren, is het essentieel om de beste praktijken te begrijpen en toe te passen om de beveiliging en betrouwbaarheid van deze systemen te waarborgen.

Ik wil graag mijn oprechte dank betuigen aan mijn mijn co-promotor, Kasper Heyndrickx, voor zijn waardevolle begeleiding en ondersteuning tijdens het schrijven van deze bachelorproef. Zijn inzichten en feedback hebben mij geholpen om mijn onderzoek naar een hoger niveau te tillen en een dieper inzicht te krijgen in het onderwerp. 

Ik wil ook Miguel De Munck, en Olivier Troch bedanken voor hun hulp en ondersteuning bij het uitvoeren van de case study. Hun expertise en advies hebben bijgedragen aan de kwaliteit van dit onderzoek.

%% Ik wil graag mijn oprechte dank betuigen aan mijn promotor, Mark Asselberg, voor zijn waardevolle begeleiding en ondersteuning tijdens het schrijven van deze bachelorproef. Zijn inzichten en feedback hebben mij geholpen om mijn onderzoek naar een hoger niveau te tillen en een dieper inzicht te krijgen in het onderwerp.