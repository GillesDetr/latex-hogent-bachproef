%%=============================================================================
%% Conclusie
%%=============================================================================

\chapter{Conclusie}%
\label{ch:conclusie}

% TODO: Trek een duidelijke conclusie, in de vorm van een antwoord op de
% onderzoeksvra(a)g(en). Wat was jouw bijdrage aan het onderzoeksdomein en
% hoe biedt dit meerwaarde aan het vakgebied/doelgroep? 
% Reflecteer kritisch over het resultaat. In Engelse teksten wordt deze sectie
% ``Discussion'' genoemd. Had je deze uitkomst verwacht? Zijn er zaken die nog
% niet duidelijk zijn?
% Heeft het onderzoek geleid tot nieuwe vragen die uitnodigen tot verder 
%onderzoek?

De combinatie van de besproken informatie biedt een diepgaand inzicht in patchmanagement, met een focus op cloudgebaseerde ERP-systemen en de rol van AI in de toekomst van patchmanagement. Het benadrukt het belang van effectieve communicatie met belanghebbenden en het gebruik van geautomatiseerde tools voor patching, vooral in een snel veranderende technologische omgeving.

Cloudgebaseerde ERP-systemen bieden aanzienlijke voordelen ten opzichte van traditionele on-premises oplossingen, waaronder schaalbaarheid, flexibiliteit en kosteneffectiviteit. De verschuiving naar de cloud is een trend die wordt gesteund door de groeiende adoptie van cloud computing-oplossingen over verschillende sectoren.

Patchmanagement in cloudgebaseerde ERP-systemen brengt echter unieke uitdagingen met zich mee, zoals het waarborgen van een tijdige implementatie van patches zonder de operationele continuïteit te verstoren. Het gebruik van geautomatiseerde patchingtools en samenwerking tussen IT-teams en cloudproviders zijn cruciaal om deze uitdagingen aan te pakken.

De toekomst van patchmanagement wordt gekenmerkt door de opkomst van AI en machine learning, die organisaties kunnen helpen bij het detecteren, prioriteren en toepassen van patches op een efficiënte manier. Hoewel het gebruik van AI veelbelovend is, brengt het ook uitdagingen met zich mee, zoals de implementatie en ethische overwegingen.

Het benadrukken van best practices, zoals het hebben van een gecentraliseerd overzicht van systemen, het plannen van patchdeployments en het testen van patches voor implementatie, is essentieel voor een effectief patchmanagementproces.

Ten slotte biedt de case study een inzichtelijke blik op de huidige situatie van patchmanagement binnen een ERP-consultancybedrijf, waarbij zowel cloudgebaseerde als on-premises oplossingen worden overwogen. Het illustreert de complexiteit van patchmanagement en benadrukt de noodzaak van geautomatiseerde tools en gestroomlijnde processen om de efficiëntie te verbeteren en de beveiliging te waarborgen.