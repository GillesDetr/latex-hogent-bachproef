%%=============================================================================
%% Conclusie
%%=============================================================================

\chapter{Conclusie}%
\label{ch:conclusie}

% TODO: Trek een duidelijke conclusie, in de vorm van een antwoord op de
% onderzoeksvra(a)g(en). Wat was jouw bijdrage aan het onderzoeksdomein en
% hoe biedt dit meerwaarde aan het vakgebied/doelgroep? 
% Reflecteer kritisch over het resultaat. In Engelse teksten wordt deze sectie
% ``Discussion'' genoemd. Had je deze uitkomst verwacht? Zijn er zaken die nog
% niet duidelijk zijn?
% Heeft het onderzoek geleid tot nieuwe vragen die uitnodigen tot verder 
%onderzoek?

Na een grondige analyse van de beschikbare tools en technologieën voor ERP-patchmanagement, interviews met relevante belanghebbenden en uitvoering van case studies bij geselecteerde bedrijven, zijn er belangrijke inzichten naar voren 
gekomen. Het onderzoek heeft aangetoond dat de behoefte aan effectief patchmanagement binnen SAP ERP-systemen van groot belang is voor organisaties, vooral gezien de toenemende dreiging van cyberaanvallen en de complexiteit van moderne IT-infrastructuren.

De resultaten tonen aan dat er verschillende overwegingen zijn bij het kiezen tussen lokale implementaties en cloudoplossingen voor ERP-systemen. Hoewel cloudoplossingen voordelen bieden zoals abonnementsmodellen en voorspelbare kosten, 
geven sommige organisaties de voorkeur aan lokale implementaties vanwege specifieke aanpassingsvereisten, dataveiligheid en compliance-eisen, behoud van bestaande investeringen en flexibiliteit en controle over configuratie.

Binnen het proces van patchmanagement is gebleken dat niet alle patches op dezelfde manier behandeld kunnen worden. Handmatige evaluatie van patches blijft vaak noodzakelijk, vooral voor kritieke CVE-patches, vanwege de variërende kostprijs,
 tijd en impact van verschillende patches. Toch kunnen tools zoals Avantra en SolarWinds Patch Manager aanzienlijke voordelen bieden bij het automatiseren en beheren van patchprocessen, afhankelijk van de specifieke behoeften van een organisatie.

Voor ERP-consultancybedrijven lijkt Avantra een sterke kandidaat te zijn vanwege zijn specialisatie in SAP-systemen en diepgaande integratie, terwijl SolarWinds Patch Manager geschikt kan zijn voor niet-SAP-systemen vanwege zijn gebruiksvriendelijke interface en eenvoudige implementatie.

Over het geheel genomen benadrukt dit onderzoek het belang van een holistische benadering van patchmanagement binnen SAP ERP-systemen. Door gebruik te maken van de juiste tools en technologieën, kunnen organisaties efficiënter omgaan met patchprocessen 
en de veiligheid en betrouwbaarheid van hun IT-infrastructuur versterken. De bevindingen van dit onderzoek bieden waardevolle inzichten en aanbevelingen voor IT-professionals en organisaties die SAP ERP-systemen gebruiken, met als doel een directe meerwaarde te bieden aan de doelgroep.
