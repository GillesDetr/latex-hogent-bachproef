%%=============================================================================
%% Conclusie
%%=============================================================================

\chapter{Conclusie}%
\label{ch:conclusie}

% TODO: Trek een duidelijke conclusie, in de vorm van een antwoord op de
% onderzoeksvra(a)g(en). Wat was jouw bijdrage aan het onderzoeksdomein en
% hoe biedt dit meerwaarde aan het vakgebied/doelgroep? 
% Reflecteer kritisch over het resultaat. In Engelse teksten wordt deze sectie
% ``Discussion'' genoemd. Had je deze uitkomst verwacht? Zijn er zaken die nog
% niet duidelijk zijn?
% Heeft het onderzoek geleid tot nieuwe vragen die uitnodigen tot verder 
%onderzoek?

Het volledige patchproces draait vooral om communicatie tussen mensen. Het kost meer tijd om afspraken te maken met klanten en een geschikt tijdslot te vinden dan om daadwerkelijk te patchen. Voor de meeste bedrijven is patchen dus geen grote klus, daarom hebben we ons gericht op bedrijven die regelmatig patches moeten uitvoeren, zoals consultancybedrijven. We hebben zelfs een case study gedaan bij zo'n bedrijf om te zien waar we het proces konden verbeteren. Uit deze studie bleek dat we het patchen zelf kunnen automatiseren met patchmanagementtools, en we hebben specifiek naar Avantra gekeken. Deze tool maakt het mogelijk om het kernel patchbestand te uploaden en met de juiste instellingen de patch uit te voeren.

Maar tools zoals Avantra zijn niet gratis, dus het is belangrijk om zorgvuldig te overwegen voordat je ze in je bedrijf integreert.

Het onderzoek biedt een diepgaand inzicht in patchmanagement, met name in cloudgebaseerde ERP-systemen en de rol van AI in de toekomst van patchmanagement. Het benadrukt het belang van goede communicatie met belanghebbenden en het gebruik van geautomatiseerde tools voor patching, vooral in een snel veranderende technologische omgeving.

Cloudgebaseerde ERP-systemen hebben veel voordelen ten opzichte van traditionele on-premises oplossingen, zoals schaalbaarheid en kosteneffectiviteit. Maar het brengt ook nieuwe uitdagingen met zich mee, zoals het zorgen voor een tijdige implementatie van patches zonder de operationele continuïteit te verstoren. Geautomatiseerde patchingtools en samenwerking tussen IT-teams en cloudproviders zijn hierbij essentieel.

De toekomst van patchmanagement ziet er spannend uit, met de opkomst van AI en machine learning die organisaties kunnen helpen bij het efficiënter detecteren, prioriteren en toepassen van patches. Hoewel het gebruik van AI veelbelovend is, zijn er ook uitdagingen, zoals de implementatie en ethische overwegingen.

Het benadrukken van best practices, zoals het hebben van een centraal overzicht van systemen en het testen van patches voor implementatie, is cruciaal voor een effectief patchmanagementproces.

Tot slot biedt de case study een kijkje in de huidige situatie van patchmanagement binnen een ERP-consultancybedrijf. Het illustreert de complexiteit van patchmanagement en benadrukt de noodzaak van geautomatiseerde tools en gestroomlijnde processen om de efficiëntie te verbeteren en de beveiliging te waarborgen.