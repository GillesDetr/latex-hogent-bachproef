%%=============================================================================
%% Conclusie
%%=============================================================================

\chapter{Conclusie}%
\label{ch:conclusie}

% TODO: Trek een duidelijke conclusie, in de vorm van een antwoord op de
% onderzoeksvra(a)g(en). Wat was jouw bijdrage aan het onderzoeksdomein en
% hoe biedt dit meerwaarde aan het vakgebied/doelgroep? 
% Reflecteer kritisch over het resultaat. In Engelse teksten wordt deze sectie
% ``Discussion'' genoemd. Had je deze uitkomst verwacht? Zijn er zaken die nog
% niet duidelijk zijn?
% Heeft het onderzoek geleid tot nieuwe vragen die uitnodigen tot verder 
%onderzoek?

Een bedrijf kan overstappen naar een cloudoplossing van een ERP-provider als het zich niet wil bezighouden met het patchproces.
In een case study bij een managed service provider werd onderzocht waar het patchproces verbeterd kon worden. Uit de studie bleek dat kernel patching geautomatiseerd kan worden met patchmanagementtools, waarbij specifiek naar Avantra werd gekeken. Deze tool maakt het mogelijk om het kernel patchbestand te uploaden en de patch met de juiste configuratie uit te voeren. Voor bedrijven die veel systemen beheren, kan dit een aanzienlijke tijdsbesparing opleveren, hoewel de voordelen pas op de lange termijn zichtbaar zullen zijn vanwege de tijd die nodig is om tools zoals Avantra te implementeren. Bovendien zijn dergelijke tools niet gratis, waardoor zorgvuldige overweging nodig is voordat men besluit ze in het bedrijf te integreren.
Het onderzoek biedt diepgaand inzicht in patchmanagement en benadrukt het belang van goede communicatie met belanghebbenden en het gebruik van geautomatiseerde tools voor patching. Cloudgebaseerde ERP-systemen bieden vele voordelen ten opzichte van traditionele on-premises oplossingen, zoals schaalbaarheid en kosteneffectiviteit. De toekomst van patchmanagement is veelbelovend, met de opkomst van AI en machine learning die organisaties kunnen helpen bij het efficiënter detecteren, prioriteren en communiceren met belanghebbenden. Hoewel het gebruik van AI veelbelovend is, zijn er ook uitdagingen, zoals de implementatie en ethische overwegingen.
Het implementeren van best practices, zoals het houden van een overzicht van systemen en het testen van patches voor implementatie, is essentieel voor een effectief patchmanagementproces. Dit onderzoek heeft belangrijke gevolgen voor de praktijk, omdat het laat zien hoe bedrijven patches kunnen uitvoeren die de betrouwbaarheid van hun ERP-systemen kunnen garanderen door ze sneller uit te voeren. De aanbevelingen stellen dat IT-teams, cloudproviders en andere belanghebbenden samen moeten werken en communiceren. Ze benadrukken ook de mogelijke noodzaak om te investeren in geautomatiseerde patchingtools voor regelmatig patchmanagement.

