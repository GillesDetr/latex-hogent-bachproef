%%=============================================================================
%% Conclusie
%%=============================================================================

\chapter{Conclusie}%
\label{ch:conclusie}

% TODO: Trek een duidelijke conclusie, in de vorm van een antwoord op de
% onderzoeksvra(a)g(en). Wat was jouw bijdrage aan het onderzoeksdomein en
% hoe biedt dit meerwaarde aan het vakgebied/doelgroep? 
% Reflecteer kritisch over het resultaat. In Engelse teksten wordt deze sectie
% ``Discussion'' genoemd. Had je deze uitkomst verwacht? Zijn er zaken die nog
% niet duidelijk zijn?
% Heeft het onderzoek geleid tot nieuwe vragen die uitnodigen tot verder 
%onderzoek?

Het volledige patchproces draait vooral om communicatie tussen stakeholders. Het kost veel meer tijd om afspraken te maken met klanten en een geschikt tijdslot te vinden dan om daadwerkelijk te patchen. Er werd een 
case study gedaan bij een managed service provider om te zien waar het proces konden verbeterd worden. Uit deze studie bleek dat het kernel patchen zelf geautomatiseerd kan worden
met patchmanagementtools, hier werd specifiek naar Avantra gekeken. Deze tool maakt het mogelijk om het kernel patchbestand te uploaden en met de juiste configuratie de patch uit te voeren. Voor bedrijven die veel systemen beheren, kan dit een tijdsbesparing opleveren, het voordeel zal echter wel maar te zien zijn op lange termijn doordat tools zoals Avantra wel wat tijd zullen innemen om te implementeren.  
Ook zijn tools zoals Avantra zijn niet gratis, dus het is belangrijk om zorgvuldig te overwegen voordat je ze in je bedrijf integreert.
Het onderzoek biedt een diepgaand inzicht in patchmanagement, het het belang van goede communicatie met belanghebbenden en het gebruik van geautomatiseerde tools voor patching wordt benadrukt. Cloudgebaseerde ERP-systemen 
hebben veel voordelen ten opzichte van traditionele on-premises oplossingen, zoals schaalbaarheid en kosteneffectiviteit. De toekomst
van patchmanagement ziet er spannend uit, met de opkomst van AI en machine learning die organisaties kunnen helpen bij het efficiënter detecteren, prioriteren en communiceren met belanghebbenden. Hoewel het gebruik van AI veelbelovend is, zijn er ook uitdagingen, zoals de implementatie en ethische overwegingen.
Het benadrukken van best practices, zoals het hebben van een centraal overzicht van systemen en het testen van patches voor implementatie, is cruciaal voor een effectief patchmanagementproces.
